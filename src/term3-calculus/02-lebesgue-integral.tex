\Subsection{Измеримые функции}
\begin{definition}
    $f: E \rightarrow \bar{\mathbb{R}}$, лебеговы мн-ва функции f:

    $E \{ f \leq a \} := \{ x \in E: \ f(x) \leq a \} = f^{-1}([-\infty, a])$

    $E \{ f < a \} := \{ x \in E: \ f(x) < a \} = f^{-1}([-\infty, a))$

    $E \{ f \geq a \} := \{ x \in E: \ f(x) \geq a \}$

    $E \{ f > a \} := \{ x \in E: \ f(x) > a \}$
\end{definition}

\begin{theorem}
    $E$ -- измеримое, $f: E \rightarrow \bar{\mathbb{R}}$, тогда равносильны:

    \begin{enumerate}
        \item {
            $E \{ f \leq a \}$ измеримы $\forall a \in \mathbb{R}$
        }
        \item {
            $E \{ f < a \}$ измеримы $\forall a \in \mathbb{R}$
        }

        \item {
            $E \{ f \geq a \}$ измеримы $\forall a \in \mathbb{R}$
        }
        \item {
            $E \{ f > a \}$ измеримы $\forall a \in \mathbb{R}$
        }
    \end{enumerate}
\end{theorem}
\begin{proof}
    \begin{enumerate}
        \item {
            $(1) \Leftrightarrow (4): $ $E\{ f > a \} = E \setminus E\{ f \leq a \}$
        }
        \item {
            $(2) \Leftrightarrow (3): $ $ E\{ f < a \} = E \setminus E\{ f \geq a \} $
        }
        \item {
            $(1) \Rightarrow (2):$ $E\{ f < a \} = \bigcup_{n=1}^{\infty} E \{ f \leq a - \frac{1}{n} \}$
        }
        \item {
            $(3) \Rightarrow (4): $ $ E \{ f > a \} = \bigcup_{n=1}^{\infty} E \{ f \geq a + \frac{1}{n} \}$
        }
    \end{enumerate}
\end{proof}

\begin{definition}
    $f: E \rightarrow \bar{\mathbb{R}}$ -- измеримая $\forall a \in \mathbb{R}$ все ее лебеговы мн-ва измер.
\end{definition}

\begin{remark}
    $E$ -- должно быть измеримое и достаточно измеримости любого множества одного типа.    
\end{remark}

\begin{example}
    \begin{enumerate}
        \item {
            $f = const$, лебеговы множества: $\emptyset, \ X$.
        }
        \item {
            $E \subset X$ -- измеримое, $f = \mathds{1}_E(x) = 1$, если $x \in E$, иначе $0$.

            Лебеговы множества: $\emptyset, X, E, X \setminus E$.
        }
        \item {
            $\mathscr{L}^m$ -- лебеговская $\sigma$-алгебра на $\mathbb{R}^m$

            $f \in C(\mathbb{R}^m)$ -- измеримая.

            $f^{-1}(\underbrace{(-\infty, a)}_{\text{измеримое}})$ -- открытое $\implies$ измеримое.
        }
    \end{enumerate}
\end{example}


\begin{properties}
    \begin{enumerate}
        \item {
            $f: E \rightarrow \bar{\mathbb{R}}$ -- измеримая $\implies E$ -- измеримое.
        }
        \item {
            Если $f: E \rightarrow \bar{\mathbb{R}}$ измеримая и $E_0 \subset E \implies g := f |_{E_0}$ -- измеримое.

            \begin{proof}
                $E_0\{ g \leq c \} = E \{ \underbrace{f \leq c}_{\text{измеримое}} \} \cap \underbrace{E_0}_{\text{измеримое}}$.
            \end{proof}
        }
        \item {
            Если $f$ -- измеримая, то прообраз любого промежутка -- измеримое мн-во.

            \begin{proof}
                $E\{ a \leq f \leq b \} = E\{ \underbrace{a \leq f}_{\text{измеримое}} \} \cap E\{\underbrace{f \leq b}_{\text{измеримое}}\}$.
            \end{proof}
        }
        \item {
            Если $f$ -- измеримая, то прообраз любого открытого мн-ва -- измеримое.

            \begin{proof}
                $U \subset \mathbb{R}$ -- открытое мн-во $\implies U = \bigcup_{n=1}^{\infty} (a_n, b_n] \implies f^{-1}(U) = \bigcup_{n=1}^{\infty} f^{-1}\underbrace{(a_n, b_n]}_{\text{измеримое}}$.
            \end{proof}
        }
        \item {
            Если $f$ -- измеримая, то $|f|$ и $-f$ -- измеримы.

            \begin{proof}
                $E \{ -f \leq c \} = E \{ f \geq -c \}, \ E \{ |f| \leq c \} = E \{ -c \leq f \leq c \}$.
            \end{proof}
        }
        \item {
            Если $f, g: E \rightarrow \bar{\mathbb{R}}$ измеримы, то $max\{ f, g \}$ и $min \{ f, g \}$ -- измеримы.

            В частности, $f_+ = max\{ f, 0 \} $ и $f_- = max\{ -f, 0 \}$ -- измеримы.

            \begin{proof}
                $E \{ max \{ f, g \} \leq c \} = E \{ f \leq c \} \cap E \{ g \leq c \}$
            \end{proof}
        }
        \item {
            Если $E = \bigcup_{n=1}^{\infty}E_n, \ f|_{E_n}$ -- измерима $\forall n \implies f$ -- измеримая.

            $f: E \rightarrow \bar{\mathbb{R}}$.

            \begin{proof}
                $E \{ f \leq c \} = \bigcup_{n=1}^{\infty} E_n \{ f \leq c \}$.
            \end{proof}
        }
        \item {
            Если $f: E \rightarrow \bar{\mathbb{R}}$ измерима, то найдется $g: X \rightarrow \bar{\mathbb{R}}$ -- измеримая, такая что $f = g|_E$

            \begin{proof}
                $g(x) := 0$, если $x \notin E, \ f(x)$, иначе.
            \end{proof}
        }
    \end{enumerate}
\end{properties}

\begin{theorem}
    Пусть $f_n: E \rightarrow \bar{\mathbb{R}}$ -- последовательность измеримых функций. Тогда:

    \begin{enumerate}
        \item $\sup{f_n}, \ \inf{f_n}$ -- измеримые.
        \item $\underline{\lim} f_n $ и $\overline{\lim} f_n$ -- измеримые.
        \item Если существуют $\lim f_n$, то он измеримый.
    \end{enumerate}
\end{theorem}

\begin{proof}
    \begin{enumerate}
        \item $E \{ \sup{f_n} \leq c \} = \bigcap_{n=1}^{\infty} E \{ f_n \leq c \}$
        \item $\underline{\lim}{f_n} = \sup_{n} \inf_{k \geq n} f_k $ и $\overline{\lim}{f_n} = \inf_{n} \sup_{k \geq n} f_k$
        \item Если существует $\lim f_n$, то $\lim f_n = \underline{\lim} f_n$.
    \end{enumerate}    
\end{proof}

\begin{theorem}
    Пусть $f_1, \dots, f_m: \ E \rightarrow H \subset \mathbb{R}$ -- измеримые,  $\phi \in C(H)$, тогда $g: E \rightarrow \mathbb{R}, \ g(x) := \phi(f_1(x), \dots, f_m(x))$ -- измеримая.
\end{theorem}
\begin{proof}
    $E \{ g < c \} = g^{-1}(-\infty, c) = \vec{f}^{-1}(U) = \vec{f}^{-1} (G)$
    
    $U := \phi^{-1}(-\infty, c)$ -- открытое в $H \implies \exists G$ --  открытое в $\mathbb{R}^{m}$, т.ч. $U = H \cap G$

    $\implies G = \bigcup_{n=1}^{\infty} \underbrace{(a_n, b_n]}_{\text{ ячейки в } \mathbb{R}^m}$

    Достаточно понять для ячейки $(\alpha, \beta]$, что $\vec{f}^{-1} (\alpha, \beta]$ -- измерима, $\bigcup_{k=1}^{n} E \{ \alpha_k < f_k \leq \beta_k \}$

    % todo: picture here
\end{proof}

\begin{consequence}
    Если в теореме $\phi$ -- поточечный предел непрерывных, то $g$ -- измерима.
\end{consequence}
\begin{proof}
    $\phi = \lim \phi_n, \ \phi_n \vec{f}$ -- измер. и поточечно стремится к $\phi_0 \vec{f}$
\end{proof}



Арифметические операции в $\mathbb{R}$:

\begin{enumerate}
    \item Если $x \in \mathbb{R}$, то $x + (+\infty) = +\infty$, $x + (-\infty) = -\infty$ и т.д.
    \item $(+\infty) +(-\infty) = 0, \ (+\infty) - (+\infty) = 0, \ (-\infty) - (-\infty) = 0$
    \item Если $0 \not = x \in \bar{\mathbb{R}}$, то $x \cdot (\pm \infty) = \pm \infty$, где знак $\pm : \pm = +, \ \pm : \mp = -$
    \item $0 \cdot \pm \infty = 0$ и $\frac{x}{\pm \infty} = 0, \ \forall x \in \bar{\mathbb{R}}$, т.е. $\frac{\pm \infty}{ \pm \infty} = 0$.
    \item Делить на 0 не умеем.
\end{enumerate}

\begin{theorem}
    \begin{enumerate}
        \item Произведение и сумма измеримых функций -- измеримая.
        \item Если  $f: E \rightarrow \mathbb{R}$ -- измеримая и $\phi \in C(\mathbb{R})$, то $\phi \circ f$ -- измеримая.
        \item Если $f \geq 0$ -- измеримая, то $f^p \ (p>0)$ -- измеримая, $(+\infty)^p = +\infty$
        \item Если $f: E \rightarrow \bar{\mathbb{R}}$ -- измеримая, $\tilde{E} := E \{f \not = 0\}$, то $\frac{1}{f}$ -- измерима на $\tilde{E}$.
    \end{enumerate}
\end{theorem}
\begin{proof}
    \begin{enumerate}
        \item {
            $f + g.$ Для каждой функции рассмотрим три множества:

            $E \{ f \not = \pm \infty \}, \ E \{ f = + \infty \}, \ E \{ f = -\infty \}$

            $E \{ g \not = \pm \infty \}, \ \underbrace{E \{ g = + \infty \}}_{= \bigcup_{n=1}^{\infty} E \{ g \geq n \}}, \ E \{ g = -\infty \}$


            Для конечного случая ($E\{f \neq \pm \infty\} \cap E\{g \neq \pm \infty\}$) можем сослаться на предыдущую теорему, взяв в качестве непрерывной $\phi(f, g) = f + g$.

            На остальных случаях тоже рассматриваем $f + g$: измеримость будет, т.к. $f + g = const$.
        }
        \item {
            Частный случай предыдущей теоремы.
        }
        \item {
            $E \{f^p \leq c\} = E \{ f \leq c^{\frac{1}{p}} \}$
        }
        \item {
            $f|_{\tilde{E}}$ -- измерима и $\not = 0$
            

            \begin{equation}
                \tilde{E}\left\{ \frac{1}{f} \leq c \right\} = 
                \begin{cases}
                    \tilde{E} \{ f \geq \frac{1}{c} \} \cup \tilde{E} \{ f < 0 \} \text{, при $c > 0$} \\
                    \tilde{E}\{ f < 0 \} \text{, при $c = 0$} \\
                    \tilde{E} \{ f \geq \frac{1}{c} \} \cap \tilde{E} \{ f < 0 \} \text{, при $c < 0$}
                \end{cases}                
            \end{equation}
        }
    \end{enumerate}
\end{proof}
\begin{consequence}
    \begin{enumerate}
        \item {
            Произведение конечного числа измер. -- измер.
        }
        \item {
            Натуральная степень измер. функции -- измер.
        }
        \item {
            Линейная комбинация измер. функций -- измер.
        }
    \end{enumerate}
\end{consequence}

\begin{theorem}
    $E \subset \mathbb{R}^{m}$ -- измеримое, $f \in C(E)$. Тогда $f$ -- измер. относительно меры Лебега.
\end{theorem}

\begin{proof}
    $U := f^{-1}(-\infty, c)$ -- открытое мн-во в $E \implies \exists G \subset \mathbb{R}^m$ --  открытое, т.ч. $U = \underbrace{G}_{\text{измер.}} \cap \underbrace{E}_{\text{измер.}}$  ($E$ измеримо по условию, а $G$ измеримо в \sigma-алгебре)
\end{proof}

\begin{definition}
    Измеримая функция -- простая, если она принимает лишь конечное число значений.

    Допустимое разбиение $X$ -- разбиение $X$ на конечное число измеримых множеств, таких что на каждом множестве простая функция константна.
\end{definition}

\begin{consequence}
    \begin{enumerate}
        \item {
            Если $X$ разбито на конечное число измер. мн-в и $f$ постоянна (то есть сужение на каждом кусочке $X$ это какая-та константа) на каждом из них, то $f$ -- простая.
        }
        \item {
            Если $f$ и $g$ -- простые функции, то у них существует общее допустимое разбиение.

            \begin{proof}
                $X = \underbrace{\bigsqcup_{k=1}^{m} A_k}_{\text{допуст. для } f} = \underbrace{\bigsqcup_{j=1}^{n} B_j}_{\text{допуст. для } g} \implies X = \bigsqcup_{k=1}^{m} \bigsqcup_{j=1}^{n} (A_k \cap B_j)$ -- допустимое для $f$ и $g$. 
            \end{proof}
        }
        \item {
            Сумма и произведение простых функций -- простая функция.
        }
        \item {
            Линейная комбинация простых функций -- простая функция.
        }
        \item {
            $\max$ и $\min$ конечного числа простых функций -- простая функция.
        }
    \end{enumerate}
\end{consequence}

\begin{theorem}
    (О приближении измеримых функций простыми)

    $f: X \rightarrow \bar{\mathbb{R}}$ -- неотрицательная измеримая функция, тогда $\exists$ последовательность простых функций $\phi_1, \phi_2 \dots$, такие что $\phi_{i} \leq \phi_{i + 1}: \ \forall i$ в каждой точке и $\lim{\phi_n} = f$. Более того, если $f$ -- ограничена сверху, то можно выбрать $\phi_n$ так, что $\phi_n \rightrightarrows f$ на $X$.
\end{theorem}
\begin{proof}
    $\Delta_k^{(n)} := [\frac{k}{n}, \frac{k+1}{n})$ при $k = 0, \dots, (n^2 - 1)$ и $\Delta_{n^2}^{(n)} := [n, +\infty]$.

    $[0, +\infty) = \bigsqcup_{k=0}^{n^2} \Delta_k, \ A_k^{(n)} := f^{-1}(\Delta_k^{(n)})$ -- измер. мн-во.

    $\phi_n$ на $A_k$ равно $\frac{k}{n} \implies 0 \leq \phi_n(x) \leq f(x) \ \forall x$ и $f(x) \leq \phi_n(x) + \frac{1}{n}$ при $x \notin A_{n^2}$.

    $\phi_n(x) \rightarrow f(x)$:

    \begin{enumerate}
        \item {
            если $f(x) = +\infty$, то $x \in A_{n^2}^{(n)} \ \forall n \implies \phi_n(x) = n \rightarrow +\infty = f(x)$
        }
        \item {
            если $f(x) \not = +\infty$, то $x \notin A_{n^2}^{(n)}$ при больших $n \implies f(x) - \frac{1}{n} \leq \phi_n(x) \leq f(x)$
        }
    \end{enumerate}

    % todo: picture here
    Для добавления монотонности берем не каждое $n$, а только степени двойки, тогда нам нужно взять $\psi_n = \max\{ \phi_1, \phi_2, \dots, \phi_n \}$ (тут должна быть картинка)

    Равномерность: если $f$ ограничена, начиная с некоторого момента $A_{n^2}$ пусто $\implies$ все $x \not \in A_{n^2} \implies \forall x \in E\ f(x) - \frac{1}{n} < \phi_n(x) \leqslant f(x) \implies |\phi_n(x) - f(x)| < \frac{1}{n} \implies$ есть равномерная сходимость.
\end{proof}



\Subsection{Последовательности измеримых функций}

Напоминание. $f_n, f : E \rightarrow \mathbb{R}$.

Поточечная сходимость: $f_n \to f$, $\forall x \in E: f_n(x) \rightarrow f(x)$

Равномерная сходимость: $f_n \rightrightarrows f$ на $E$, $\sup_{x \in E} |f_n(x) - f(x)| \rightarrow 0$

\begin{definition}
    $f_n, f: E \rightarrow \mathbb{R}$ -- измеримые.

    $f_n$ сходится к $f$ \textbf{почти везде}, если $\exists e \subset E, \ \mu e = 0$, т.ч. $\forall x \in E \setminus e, \ f_n(x) \rightarrow f(x)$

    \begin{remark}
        Обозначение: $\mathscr{L}(E, \mu) = \{ f: E \rightarrow \overline{\mathbb{R}} - \text{ измеримые, } \mu E\{ f = \pm \infty\} = 0\}$
    \end{remark}

    Пусть $f_n, f \in \mathscr{L}(E, \mu), \ f_n$ сходится к $f$ почти везде.

    $\exists e \subset E, \ \mu e = 0, $ т.ч. $\forall x \in E \setminus x, \ f_n(x) \rightarrow f(x)$
\end{definition}


\begin{definition}
    $f_n, f \in \mathscr{L}(E, \mu), \ f_n$ сходится по мере $\mu$ к $f$, если $\forall \varepsilon > 0, \\ \mu E \{ | f_n - f | > \varepsilon \} \rightarrow_{n \rightarrow \infty} 0, \ f_n \Rightarrow_{\mu} f$
\end{definition}

\begin{remark}
    Зависимость: равномерная $\implies$ (поточечная $\implies$ почти везде) | (сходимость по мере).

    Равномерная $\implies$ поточечная -- знаем.

    Поточечная $\implies$ почти везде -- у нас уже есть сходимость во всех точках, поэтому для ``почти везде'' ничего не надо выкидывать.

    Равномерная $\implies$ сходимость по мере -- начиная с некоторого момента $E\{|f_n - f| > \varepsilon \}$ будет пустым множеством по определению равномерной сходимости.
\end{remark}

\begin{statement}
    \begin{enumerate}
        \item {
            Если $f_n$ сходится к $f$ п.в. (почти везде) и $f_n$ сходится к $g$ п.в., то $f = g$ (за исключением мн-ва нулевой меры)
        }
        \item {
            Если $f_n \Rightarrow_{\mu} f$ и $f_n \Rightarrow_{\mu} g$, то $f = g$ за исключением мн-ва нулевой меры.
        }
    \end{enumerate}
\end{statement}
\begin{proof}
    \begin{enumerate}
        \item {
            Берем $e \subset E, \ \mu e = 0$ и $\lim{f_n(x)} = f(x), \ \forall x \in E \setminus e$

            $\tilde{e} \subset E, \mu \tilde{e} = 0$ и $\lim{f_n(x)} = g(x), \ \forall x \in E \setminus \tilde{e}$
        
            Тогда на $E \setminus (e \cup \tilde{e}) \lim f_n(x) = g(x)$ и $\lim f_n(x) = f(x) \implies f(x) = g(x) \forall x \in E \setminus (e \cup \tilde{e})$ 
        }
        \item {
            $\mu E \{ f \not = g \} \underbrace{=}_{?} 0, \ E\{ f \not = g \} = \bigcup_{k=1}^{\infty} E \{ |f - g| > \frac{1}{k} \}$.

            Достаточно доказать, что $\mu E \{ |f - g| \geq \epsilon \} = 0$.

            $E \{ |f - g| \geq \epsilon \} \subset E \{ |f_n - f| \geq \frac{\epsilon}{2} \} \cup E \{ |f_n - g| \geq \frac{\epsilon}{2} \}$


            $E \{ |f - g| \geq \epsilon \} \subset \underbrace{\bigcap_{n=1}^{\infty} E \{ |f_n - f| \geq \frac{\epsilon}{2} \}}_{\mu = 0 \ ?} \cup \bigcap_{n=1}^{\infty} E \{ |f_n - g| \geq \frac{\epsilon}{2} \}$

            Знаем, что $\mu E \{ |f_n - f| \geq \frac{\epsilon}{2} \} \rightarrow 0$

            $\bigcap_{n=1}^{N} E \{ |f_n - f| \geq \frac{\epsilon}{2} \}$ вложены по убыванию

            $\implies \bigcap_{n=1}^{\infty} \dots = \lim_{N} \left( { \mu \bigcap_{n=1}^{N} E \{ |f_n - f| \geq \frac{\epsilon}{2} \} } \right) \leq \lim_{N} \left( {\mu E \{ |f_N - f| \geq \frac{\epsilon}{2} \}}\right) = 0$
        }
    \end{enumerate}
\end{proof}

\begin{theorem}
    \textbf{Лебега}.

    $f_n, f \in \mathscr{L}(E, \mu)$

    Пусть $\mu E < +\infty$ и $f_n$ сходится к $f$ почти везде.

    Тогда $f_n$ сходится к $f$ по мере $\mu$.
\end{theorem}
\begin{proof}
    Найдется $e \subset E, \ \mu e = 0$, т.ч. $\forall x \in \subset E \setminus e, \ f_n(x) \rightarrow f(x)$.

    Выкинем $e$ и будем говорить про поточечную сходимость.

    Надо доказать, что $A_n := E \{ |f_n - f| > \epsilon \}, \ \mu A_n \rightarrow 0$.

    \begin{enumerate}
        \item {
            Частный случай ($f_n \searrow 0$): $A_n = E \{ f_n > \epsilon \} \supset A_{n+1}$.

            $\lim{\mu A_n} = \mu \bigcap_{n=1}^{\infty} A_n = \mu \emptyset = 0$.
            
            Пусть $x \in \bigcap_{n=1}^{\infty} A_n \implies 0 \leftarrow f_n(x) > \epsilon \ \forall n \in \mathbb{N} \implies $ таких $x$ не существует.
        }
        \item {
            Общий случай: $g_n(x) := \sup_{k \geq n} \{ |f_k(x) - f(x)| \}$. $g_n(x) \searrow$, т.к. множество уменьшается.

            $\lim {g_n(x)} = \lim_{n} \sup_{k \geq n} \{ \dots \} = \overline{\lim_n {|f_n(x) - f(x)|}} = \lim {|f_n - f|} = 0$

            $\implies \underbrace{\mu E \{ g_n > \epsilon \}}_{\rightarrow 0} \geq \mu E \{ |f_n - f| > \epsilon \}$

            $E \{ g_n > \epsilon \} \supset E \{ |f_n - f| > \epsilon \}$
        }
    \end{enumerate}
\end{proof}


\begin{remark}
    \begin{enumerate}
        \item {
            Условие $\mu E < +\infty$ существенно.

            $E = \mathbb{R}, \ \mu = \lambda, \ f_n = \mathds{1}_{[n, +\infty)} \underbrace{\rightarrow}_{\text{поточечно}} f \equiv 0$

            $\lambda E \{ f_n > \epsilon \} = +\infty \not \rightarrow 0$.
        }
        \item {
            Обратное неверно. Более того, может быть сходимость по мере и расходимость во всех точках вообще: $E = [0, 1), \ \mu = \lambda$

            $\mathds{1}_{[0, 1)} \ \mathds{1}_{[0, \frac{1}{2})} \ \mathds{1}_{[\frac{1}{2}, 1)} \ \mathds{1}_{[0, \frac{1}{3})} \ \mathds{1}_{[\frac{1}{3}, \frac{2}{3})} \ \mathds{1}_{[\frac{2}{3}, 1)}$ -- ни для какого аргумента нет предела: $[0, \frac{1}{n}) \ [\frac{1}{n}, \frac{2}{n}) \dots [\frac{n - 1}{n}, 1)$
        }
    \end{enumerate}
\end{remark}


\begin{theorem}
    \textbf{Рисса}.

    $f, f_n \in \mathscr{L}(E, \mu)$. Если $f_n \Rightarrow_{\mu} f$, то существует подпоследовательность $f_{n_k}$, т.ч. $f_{n_k}$ сходится к $f$ почти везде.
\end{theorem}
\begin{proof}
    $\mu E \{ |f_n - f| > \frac{1}{k} \} \underbrace{\rightarrow}_{n \rightarrow \infty} 0$

    Выберем $n_k$ так, что $n_k > n_{k - 1},$ и $\mu \underbrace{E \{ |f_{n_k} - f| > \frac{1}{k} \}}_{=: A_k} < \frac{1}{2^k}$

    $B_n := \bigcup_{k=n}^{\infty} A_k, \ \mu B_n \leq \sum_{k=n}^{\infty} \mu A_k < \sum_{k=n}^{\infty} \frac{1}{2^k} = \frac{1}{2^{n - 1}} \rightarrow 0$

    $B_1 \supset B_2 \supset \dots \implies \underbrace{\mu B}_{\mu B_n \rightarrow 0} = 0$, проверим, что если $x \notin B$, то $f_{n_k}(x) \rightarrow f(x)$, где $B := \bigcap_{n=1}^{\infty} B_n$

    $x \notin B \implies \exists m$, т.ч. $x \notin B_m = \bigcup_{k=m}^{\infty} A_k$

    $\implies x \notin A_k \ \forall k \geq m \implies \forall k \geq m \ \underbrace{|f_{n_k}(x) - f(x)|}_{\rightarrow_{k \rightarrow 0} 0} \leq \frac{1}{k}$
\end{proof}

\begin{consequence}
    Если $f_n \leq g$ и $f_n {\Rightarrow}_{\mu} f$, то $f \leq g$ за исключением мн-ва нулевой меры.
\end{consequence}
\begin{proof}
    Выберем $f_{n_k}$ сходится к $f$ почти везде. Пусть $e$ -- исключ. мн-во $\mu e = 0$.

    $\lim \underbrace{f_{n_k}}_{\leq g(x)} = f(x): \ \forall x \in E \setminus e \implies f(x) \leq g(x)$ при $x \in E \setminus e$
\end{proof}

\begin{theorem}
    \textbf{Фреше}.

    Если $f: \mathbb{R}^{m} \rightarrow \mathbb{R}$ измерима относительно $\lambda_m$ (мера Лебега), то $\exists f_n \in C(\mathbb{R}^m)$, т.ч. $f_n$ сходится к $f$ почти везде.
\end{theorem}


\begin{theorem}
    \textbf{Егорова}.

    Пусть $\mu E < + \infty, \ f_n, f \in \mathscr{L}(E, \mu)$. Если $f_n$ сходится к $f$ почти везде, то найдется $e \subset E, \ \mu e < \epsilon$, т.ч. $f_n \rightrightarrows f$ на $E \setminus e$.
\end{theorem}

\begin{theorem}
    \textbf{Лузина}.

    $E \subset \mathbb{R}^m$ -- измеримо, $f: E \rightarrow \mathbb{R}$  -- измерима (относительно $\lambda_m$ -- мера Лебега). Тогда найдется $e \subset E, \ \mu e < \epsilon$, т.ч. $f|_{E \setminus e}$ -- непрерывна.


    Фреше + Егоров $\implies$ Лузин:

    $f: \mathbb{R}^m \rightarrow \mathbb{R}$ -- измеримое $\underbrace{\implies}_{\text{Фреше}} \exists f_n \in C(\mathbb{R}^m)$, $f_n$ сходится к $f$ почти везде $\underbrace{\implies}_{\text{Егоров}}$ $\exists e: \ \lambda_m e < \epsilon$, т.ч. $f_n \underbrace{\rightrightarrows}_{\mathbb{R}^m \setminus e} f$, равномерный предел непрерывной функции -- непрерывная функция.
\end{theorem}

\Subsection{Определение интеграла}

\begin{lemma}
    Пусть $f \geq 0$ простая функция $A_1, \dots, A_n$ и $B_1, \dots, B_m$ -- допустимые разбиения.

    $a_1, \dots, a_n$ и $b_1, \dots, b_m$ значения $f$ на соответственных мн-вах.

    Тогда $\sum_{k=1}^{n} a_k \mu (E \cap A_k) = \sum_{j=1}^{m} b_j \mu (E \cap B_j)$.
\end{lemma}

\begin{proof}
    $\sum_{k=1}^{n} a_k \mu (E \cap A_k) = \sum_{k=1}^{n} \sum_{j=1}^{m} a_k \mu (E \cap A_k \cap B_j) = (1)$

    $\sum_{j=1}^{m} b_j \mu (E \cap B_j) = \sum_{j=1}^{m} \sum_{k=1}^{n} b_j \mu (E \cap B_j \cap A_k) = (2)$

    $(1) \underbrace{=}_{?} (2)$.

    $a_k \mu (E \cap A_k \cap B_j) = b_j \mu (E \cap A_k \cap B_j)$

    если $A_k \cap B_j \not = \emptyset$, то $a_k = b_j$, если $A_k \cap B_j = \emptyset$, то $\mu (\dots) = 0$.

    Условие $f \geq 0$ важно, т.к. в ином случае могли бы получится $\infty$ разных знаков и равенство зависело бы от порядка сложения.
\end{proof}

\begin{definition}
    $f \geq 0$ простая, $\int_{E} f d \mu := \sum_{k=1}^{n} a_k \mu (E \cap A_k)$, где $A_1, \dots , A_n$ -- допустимые разбиения ($\bigsqcup_{k=1}^{n} A_k = X$), $a_1, \dots, a_n$ -- соответст. значения.
\end{definition}

\begin{properties}
    \begin{enumerate}
        \item {
            $\int_{E} c d \mu = c \mu E, \ c \geq 0$
        }
        \item {
            Если $f, g$ -- простые и $0 \leq f \leq g$, то $\int_{E} f d \mu \leq \int_E g d \mu$
        }
        \item {
            Если $f, g \geq 0$ -- простые, то $\int_{E} (f+g) d \mu = \int_{E} f d \mu + \int_{E} g d \mu$
        }
        \item {
            Если $c \geq 0$ и $ f \geq 0$ -- простая, то $\int_{E} c f d \mu = c \cdot \int_{E} f d \mu$
        }
    \end{enumerate}
\end{properties}
\begin{proof}

    $\bigsqcup_{k=1}^{n} A_k = X$ -- общее допустимиое разбиение, $a_k, b_k$ -- значения на $A_k$.


    3. $\int_{E} (f + g) d \mu = \sum (a_k + b_k) \mu (E \cap A_k) = \sum a_k \mu (A_k \cap E) + \sum b_k \mu (A_k \cap E) = \int_E d f \mu + \int_E g d \mu$
    
    2. $\int_E f d \mu = \sum a_k \mu (A_k \cap E) \leq \sum b_k \mu (A_k \cap E) = \int_E g d \mu$
\end{proof}

\begin{definition}
    Интеграл от неотриц. измеримой ф-ции $f: E \to \overline{R}, f \geq 0$.

    $\int_E f d \mu := \sup \{ \int_E \phi d \mu : \ \phi \text{ -- простая и } 0 \leq \phi \leq f \}$
\end{definition}

\begin{definition}
    Интеграл от измеримой функции

    $\int_E f d \mu := \int_E f_+ d \mu - \int_E f_- d \mu$ (если тут $+\infty - (+\infty)$, то интеграл не определен)
\end{definition}

\begin{remark}
    Новое определение на простых функциях совпадает со старым.

    \begin{proof}
        $f \geq 0$ -- простая $ \implies$
        
        (1): $\phi = f$ подходит (новое $\geq$ старое, т.к. берем супремум).
    
        (2): $\phi \leq f \implies \int_E \phi d \mu \leq \int_E f d \mu$ ($\sup \leq$ старое, т.к. задали $\phi: 0 \leqslant \phi \leqslant f$).

        (3): В определении для произвольных измеримых: $\int_E (f)_{-}d\mu = 0$
    \end{proof}
\end{remark}

\begin{properties}
    \begin{enumerate}
        \item Если $0 \leq f \leq g \implies \int_E f d \mu \leq \int_E g d \mu$
        \item Если $\mu E = 0 \implies \int_E f d \mu = 0$
        \item {
            $f$ -- измеримая $\implies \int_E f d \mu = \int_X \mathds{1}_E f d \mu$ 

            \begin{proof}
                Проверим для $f_{\pm}$:
                
                $\int_E f_+ d \mu = \sup \{  \int_E \phi d \mu: \ \phi \text{ -- простая } 0 \leq \phi \leq f_+ \} = \sup \{ \int_X \phi d \mu: \phi \text{ -- простая } 0 \leq \phi \leq \mathds{1}_E f_+ \} = \int_X \mathds{1}_E f_+ d \mu$ (в одном случае сужаем $\phi$ на множество $E$, в другом -- дополняем нулями на $X \setminus E$)
            \end{proof}
        }
        \item {
            Если $f \geq 0$ -- измеримая, $A \subset B$, то $\int_A f d \mu \leq \int_B f d \mu$.

            \begin{proof}
                $\int_A f d \mu = \int_X \mathds{1}_A f d \mu \underbrace{\leq}_{\text{т.к. } \mathds{1}_A f \ \leq \ \mathds{1}_B f} \int_X \mathds{1}_B f d \mu = \int_B f d \mu$.
            \end{proof}
        }
    \end{enumerate}
\end{properties}

\begin{exerc}
    Доказать, что $\int_{[1; +\infty)}{\frac{\sin{x}}{x} d \lambda_1}$ не определен.
\end{exerc}

% todo: change \mathds{1} to smth that looks more like 'fat one'? wtf

\begin{theorem}
    \textbf{Беппо Леви}.

    Пусть $f_n \geq 0$ -- измеримые функции, $f_n: E \to \overline{R}$, последовательность поточечно возрастающая $f_0 \leq f_1 \leq f_2 \leq \dots$. $f(x) := \lim{f_n(x)}$ -- поточечный предел.

    Тогда $\int_E {f d \mu} = \lim{\int_E {f_n d \mu}}$.
\end{theorem}
\begin{proof}
    (1): $f_n \leq f \implies \int_E {f_n d \mu} \leq \int_E{f d \mu}$

    (2): $f_n \leq f_{n+1} \implies \int_E{f_n d \mu} \leq \int_E{f_{n+1} d \mu}$

    (1) и (2) $\implies \exists L := \lim{\int_E{f_n d \mu}} \leq \int_E{f d \mu}$

    Осталось проверить, что $L \geq \int_E{f d \mu}$ (можно считать, что $L < +\infty$ т.е. конечна, иначе утверждение очевидно).

    $\int_E {f d \mu} = \sup \{ \int_E {\phi d \mu}: \ 0 \leq \phi \leq f, \ \phi \text{ -- простая} \}$
    
    Достаточно доказать, что $L \geq \int_E{\phi d \mu}$ для $\phi$ -- простая и $0 \leq \phi \leq f$.

    Возьмем $0 < \theta < 1$ и докажем, что $L \geq \int_E {\theta \phi d \mu}$:

    $E_n := E \{ f_n \geq \theta \phi \}, f_n \nearrow \implies E_n \subset E_{n+1}$. Покажем, что $E = \bigcup_{n=1}^{\infty} E_n$.

    Пусть $x \in E$:
    
    \begin{enumerate}
        \item если $\phi(x) = 0$, то $\forall n : \ x \in E_n$
        \item если $\phi(x) > 0$, то $\lim{f_n(x)} = f(x) \geq \phi(x) > \theta \phi(x) \underbrace{\implies}_{\text{при больших }n} f_n(x) > \theta \phi(x) \underbrace{\implies}_{\text{при больших }n} x \in E_n$
    \end{enumerate}


    Посмотрим на $\underbrace{\int_E{f_n d \mu}}_{(*)} \geq \int_{E_n}{f_n d \mu} \geq \underbrace{\int_{E_n}{\theta \phi d \mu}}_{(**)}$.

    Переходим к пределу $n \rightarrow \infty: $ $\underbrace{L}_{\text{получили из }(*)} \geq \underbrace{\int_E {\theta \phi d \mu}}_{\text{это нужно понять для }(**)}$

    Осталось понять, что $\underbrace{\int_{E_n} {\phi d \mu}}_{\sum_{k=1}^{m} a_k \mu (E_n \cap A_k)} \rightarrow \underbrace{\int_E {\phi d \mu}}_{\sum_{k=1}^{m} \mu (E \cap A_k)}$.

    Поймем, что $\mu (E_n \cap A_k) \rightarrow \mu (E \cap A_k)$ -- непрерывность меры снизу, $E_n \cap A_k \subset E_{n+1} \cap A_k$ и $\bigcup_{k=1}^{\infty} (E_n \cap A_k) = E \cap A_k$.
\end{proof}

\begin{properties}
    Продолжаем писать свойства:

    5. $f, g \geq 0$ -- измеримые $\implies \int_E{(f+g) d \mu} = \int_Ef d \mu + \int_E{g d \mu}$ -- аддитивность.

    6. $f \geq 0, \alpha \geq 0 \implies \int_E{\alpha f d \mu} = \alpha \int_E{f d \mu}$ -- однородность.

    7. $\alpha, \beta \geq 0, \ f, g \geq 0$ -- измеримые, тогда $\int_E{(\alpha f + \beta g) d \mu} = \alpha \int_E f d \mu + \beta \int_E g d \mu$
    %todo: кто понимает, что писать вместо \dots напишите, плиз
    %это вроде просто линейная комбинация, но я оставлю тудушку на всякий случай, а то вдруг я банан
\end{properties}

\begin{proof}
    5. $f \geq 0$ измеримая $\implies \exists 0 \leq \phi_1 \leq \phi_2 \leq \dots$ -- простые, причем $\phi_n \rightarrow f$ поточечно.
    
    $g \geq 0$ измеримая $\implies \exists 0 \leq \psi_1 \leq \psi_2 \leq \dots$ -- причем $\psi_n \rightarrow g$ поточечно.

    $\implies 0 \leq \phi_1 + \psi_1 \leq \dots$ простые и $\phi_n + \psi_n \rightarrow f + g$.


    $\underbrace{\int_E{(\phi_n + \psi_n) d \mu}}_{\rightarrow \int_E{(f+g) d \mu}} = \underbrace{\int_E{\phi_n d \mu}}_{\underbrace{\rightarrow}_{\text{по Леви}} \int_E d \mu} + \underbrace{\int_E{\psi_n d \mu}}_{\rightarrow \int_E{g d \mu}}$
\end{proof}

\begin{properties}
    Продолжаем свойства.

    8. Аддитивность по мн-ву. Если $A \cap B = \emptyset, \ f \geq 0$ измеримая, то 
    $\underbrace{\int_{A\cup B}{f d \mu}}_{(*)} = \underbrace{\int_A{f d \mu}}_{(**)} + \underbrace{\int_B{f d \mu}}_{(***)}$

    \begin{proof}
        $(*) = \int_X{\mathds{1}_{A \cup B} f d \mu}$

        $(**) = \int_X{\mathds{1}_{A} f d \mu}$

        $(***) = \int_X{\mathds{1}_{B} f d \mu}$

        $\mathds{1}_{A \cup B} f = \mathds{1}_A f + \mathds{1}_B f$
    \end{proof}

    9. Если $\mu E > 0$ и $f > 0$ измери., то $\int_E{f d \mu} > 0$.

    \begin{proof}
        $E_n := E \{ f \geq \frac{1}{n} \}, \ E_n \subset E_{n+1}, \ E = \bigcup_{n=1}^{\infty} E_n$

        $\implies \lim{\mu E_n} = \mu E > 0 \implies \mu E_n > 0$ для больших $n$

        $\implies \int_E{f d \mu} \geq \int_{E_n} f d \mu \geq \int_{E_n} {\frac{1}{n} d \mu} = \frac{1}{n} \cdot \mu E_n > 0$.
    \end{proof}
\end{properties}

\begin{example}
    $T = \{ t_1, t_2, \dots \}$ - не более чем счетное, $w_1, w_2, \dots \geq 0$.

    $\mu A := \sum_{k: \ t_k \in A}{w_k}$ -- мера.

    $\int_E{f d \mu} = \sum_{k: \ t_k \in E}{w_k} = (*)$.

    Пусть $f = \mathds{1}_A, $ тогда $\int_E{f d \mu} = \int_E{\mathds{1}_A d \mu} = \mu (E \cap A) = \sum_{k: \ t_k \in E \cap A} = \sum_{k: \ t_k \in E}{\mathds{1}(t_k) w_k} = (*)$. \newline

    $\implies$ равенство есть и на простых функциях

    Пусть $f \geq 0$ измерим. $\phi_n = f \cdot \mathds{1}_{\{ t_1, t_2, \dots, \phi_n \}}$, $0 \leq \phi_1 \leq \dots \leq f$.

    $\underbrace{\lim{\int_E{\phi_n d \mu}}}_{= \lim \sum_{k<n: \ t_k \in E}{f(t_k) w_k} = \sum_{k: \ t_k \in E}{f(t_k) w_k}} = \int_E{\underbrace{\lim{\phi_n}}_{\leq f} d \mu} \leq \int_E {f d \mu}$

    Проверим, что $\underbrace{\int_E{f d \mu}}_{\sup \{ \dots\}} \leq \sum_{f_(t_k)w_k}$. Берем $0 \leq \underbrace{\phi}_{\text{простая}} \leq f$ и проверяем, что $\underbrace{\int_E{\phi d \mu}}_{\sum_{k: \ t_k \in E}{\phi(t_k) w_k}} \leq \sum_{k: \ t_k \in E}{f(t_k)w_k}$
\end{example}
\begin{remark}
    $T = \mathbb{N}, \ w_n \equiv 1$.

    $\mu A = \#\{ A \cap \mathbb{N} \}$

    $\int_{\mathbb{N}}{f d \mu} = \sum_{n=1}^{\infty} f(n)$
\end{remark}

\begin{definition}
    $P(x)$ -- св-во, зависящее от точки. $P(x)$ выполняется \textbf{почти везде}, если на $E$ (для \textbf{почти всех} точек из $E$), если $\exists e \subset E, \ \mu e = 0$ и $P(x)$ выполнено $\forall x \in E \setminus e$.
\end{definition}

\begin{remark}
    $P_1, P_2, \dots$ последовательность св-в, каждое из котороых верно почти везде на $E$, то они все вместе верны почти везде на $E$.
\end{remark}

\begin{theorem}
    (Неравенство Чебышева).
    
    $f \geq 0$ измер., $t, p > 0$. Тогда $\mu E \{ f \geq t \} \leq \frac{1}{t^p} \cdot \int_E{f^p d \mu}$.
\end{theorem}
\begin{proof}
    $\int_E{f^p d \mu} \geq \int_{E\{ f \geq t \}}{f^p d \mu} \geq \int_{E\{ f \geq t \}}{t^p d \mu} = t^p \cdot \mu E \{ f \geq t \}$.
\end{proof}

\begin{properties}
    Свойства интеграла, связанные с понятием "почти везде".

    \begin{enumerate}
        \item Если $\int_{E}{|f| d \mu} < + \infty$, то $f$ почти везде конечна.
        \item Если $\int_E{|f| d \mu = 0}$, то $f = 0$ почти везде.
        \item Если $A \subset B$ и $\mu (B \setminus A) = 0$, то $\int_A{f d \mu}$ и $\int_B{f d \mu}$ либо определены, либо нет одновременно. И если определены, то равны.
        \item Если $f = g$ почти везде на $E$, тогда $\int_E{f}$ и $\int_E{g}$ либо определены, либо нет одновременно. И если определены, то равны.
    \end{enumerate}
\end{properties}

\begin{proof}
    \begin{enumerate}
        \item {
            $E\{ |f| = +\infty \} \subset E\{ |f| \geq t \}$

            $\mu E \{ |f| = +\infty \} \leq \mu E \{ |f| \geq t \} \leq \frac{\int_E{|f| d \mu}}{t} \underbrace{\rightarrow}_{t \rightarrow +\infty} 0$
        }
        \item {
            Если $\mu E \{ f > 0 \} > 0$, то $\int_E{f d \mu} = \int_{E\{ f > 0 \}}{f d \mu} > 0$ (св-во. 9 из уже доказанных выше).
        }
        \item {
            $\int_B{f_{\pm} d \mu} = \int_{B \setminus A}{f_{\pm} d \mu} + \int_{A}{f_{\pm} d \mu} = \int_A{f_{\pm} d \mu}$
        }
        \item {
            $A := E\{ f = g \}, \mu (E \setminus A) = 0 \ \int_{E}{f d \mu} = \int_{A}{f d \mu} = \int_{A}{g d \mu} = \int_E{g d \mu}$
        }
    \end{enumerate}
\end{proof}

\Subsection{Суммируемые функции}

\begin{definition}
    $f$ -- суммируема на мн-ве $E$, если $f$ измерима и $\int_E{f_{\pm} d \mu} < +\infty$.    
\end{definition}
\begin{remark}
    В этом случае $\int_E{f d \mu}$ конечен.
\end{remark}
\begin{properties}
    \begin{enumerate}
        \item {
            $f$ -- суммируема на $E \Leftrightarrow \int_E{|f| d \mu} < +\infty$ и $f$ -- измерима.

            В этом случае $|\int_E{f d \mu}| \leq \int_E{|f| d \mu}$

            \begin{proof}
                $0 \leq f_{\pm} \leq |f| = f_+ + f_-$

                "$\Rightarrow$":  $\int_E{|f| d \mu} = \int_E{f_+ d \mu} + \int_E{f_- d \mu} < +\infty$

                "$\Leftarrow$": $\int_E{f_{\pm} d \mu} \leq \int_{E}{|f|d \mu} < +\infty$


                Нер-во: $-\int_E{|f| d \mu} = -\int_E{f_+ d \mu} - \int_E{f_- d \mu} \leq \underbrace{\int_E{f_+ d \mu} - \int_E{f_- d \mu}}_{\int_E{f d \mu}} \leq \int_E{f_+ d \mu} + \int_E{f_- d \mu} = \int_E{|f| d \mu}$
            \end{proof}
        }
        \item {
            $f$ суммируема на $E \implies f$ почти везде конечна на $E$.
        }
        \item {
            Если $A \subset B$ и $f$ суммируема на $B$, то $f$ суммируема на $A$.

            \begin{proof}
                $\int_A{|f| d \mu} \leq \int_B{|f| d \mu} < +\infty$
            \end{proof}
        }
        \item {
            Ограниченная функция суммируема на мн-ве конечной меры.

            \begin{proof}
                $|f| \leq M \implies \int_E{|f| d \mu}  \leq \int_E{M d \mu} = M \cdot \mu E < +\infty$
            \end{proof}
        }
        \item {
            Если $f$ и $g$ суммируемы и $f \leq g$, то $\int_E{f d \mu} \leq \int_E{g d \mu}$

            \begin{proof}
                $f_+ - f_- = f \leq g = g_+ - g_- \implies 0 \leq f_+ + g_- \leq f_- + g_+ \implies \int_E{f_+ d \mu} + \int_E{g_- d \mu} \leq \int_E{f_- d \mu} + \int_E{g_+ d \mu}$ -- переносим слагаемые в нужные стороны и чтд.
            \end{proof}
        }
        \item {
            $f$ и $g$ -- суммируемы $\implies f + g$ суммируема и $\int_E{(f+g) d \mu} = \int_E{f d \mu} + \int_E{g d \mu}$

            \begin{proof}
                $|f + g| \leq |f| = |g| \implies f+g$ суммируема.

                $h := f + g, \ h_+ - h_- = f_+ -  f_- + g_+ - g_-$

                $\implies h_+ + f_- + g_- = f_+ + g_+ + h_- \geq 0$

                $\implies \int_E{h_+ d \mu} + \int_E{f_- d \mu} + \int_E{g_- d \mu} = \int_E{f_+ d \mu} + \int_E{g_+ d \mu} + \int_E{h_- d \mu}$ -- далее просто переносим нужные слогаемые через равно.
            \end{proof}
        }
        \item {
            $f$ -- суммируема, $\alpha \in \mathbb{R} \implies \alpha f$ суммируема и $\int_E{\alpha f d \mu} = \alpha \int_E{f d \mu}$

            \begin{proof}
                $|\alpha f| = |\alpha| \cdot |f| \implies |\alpha f|$ -- суммируема.

                Если $\alpha > 0$, то $(\alpha f)_+ = \alpha \cdot f_+$ и $(\alpha f)_- = \alpha \cdot f_-$ и $\int_E{(\alpha f)_{\pm} d \mu} = \alpha \cdot \int_E{f_{\pm} d \mu}$ 

                Если $\alpha = -1$, то $(-f)_+ = f_-$ и $(-f)_- = f_+ \implies \int_E{(-f) d \mu} = \int_E{f_-} - \int_E{f_+} = - \int_E{f d \mu}$
            \end{proof}
        }
        \item {
            Линейность. 

            Если $f, g$ -- суммируемы, $\alpha, \beta \in \mathbb{R}$, то $\alpha f + \beta g$ -- суммируема и $\int_E{(\alpha f + \beta g) d \mu} = \alpha \int_E{f d \mu} + \beta \int_E{g d \mu}$.
        }
        \item {
            Пусть $E = \bigcup_{k=1}^{n} E_k$. Тогда $f$ -- суммируема на $E \Leftrightarrow f$ -- суммируема на $E_k: \ \forall k = 1, \dots, n$. А если $f$ суммируема на $E = \bigsqcup_{k=1}^{n} E_k$, то $\int_E{f d \mu} = \sum_{k=1}^{n} \int_{E_k}{f d \mu}$

            \begin{proof}
                $\mathds{1}_{E_k} |f| \leq \mathds{1}_E |f| \leq \sum_{k=1}^{n} \mathds{1}_{E_k} |f| \implies \int_{E_k}{|f| d \mu} \leq \sum_{k=1}^{n} \int_{E_k}{|f| d \mu}$.
                
                Если $E = \bigsqcup_{k=1}^{n} E_k$, то $\mathds{1}_E = \sum_{k=1}^{n} \mathds{1}_{E_k} \implies \mathds{1}_E f_{\pm} = \sum_{k=1}^{n} \mathds{1}_{E_k} f_{\pm} \implies \int_E{f_{\pm} d \mu} = \sum_{k=1}^{n} \int_{E_k}{f_{\pm} d \mu}$
            \end{proof}
        }
        \item {
            Интегрирование по сумме мер. Пусть $\mu_1$ и $\mu_2$ -- меры, заданные на одной $\sigma$-алгебре, $\mu := \mu_1 + \mu_2$.

            Если $f \geq 0$ измерима, то $\int_E{f d \mu} = \int_E{f d \mu_1} + \int_E{f d \mu_2} (*)$.

            $f$ -- суммируема относительно $\mu \Leftrightarrow f$ -- суммируема относительно $\mu_1$ и $\mu_2$ и в этом случае есть равенство $(*)$.

            \begin{proof}
                $(*)$ для $f \geq 0$: 
                
                $(*)$ есть для простых $\phi \geq 0$, $\int_E{\phi d \mu} = \sum_{k=1}^{n} a_k \underbrace{\mu(E \cap A_k)}_{\mu_1(E \cap A_k) + \mu_2(E \cap A_k)} = \int_E{\phi d \mu_1} + \int_E{\phi d \mu_2}$.

                $f \geq 0$ -- измеримая $\implies$ возьмем $0 \leq \phi \leq \dots \leq \phi_n$ -- простые, $\phi_n \rightarrow f$.

                $\int_E{\phi_n d \mu} = \int_E{\phi_n d \mu_1} + \int_E{\phi_n d \mu_2}$ по т. Леви получаем (предельнй переход) $\int_E{f d \mu} = \int_E{f d \mu_1} + \int_E{f d \mu_2}$
            \end{proof}
        }
    \end{enumerate}
\end{properties}


\begin{definition}
    Интеграл от комплекснозначной функции $f: E \rightarrow \mathbb{C}$. 

    $Re (f)$ и $Im (f)$ -- измеримые функции.

    $\int_E{f d \mu} := \int_E{Re(f) d \mu} + i \cdot \int_E{Im(f) d \mu}$
\end{definition}
\begin{remark}
    Все св-ва, связанные с равенствами, сохраняются:
\end{remark}
\begin{proof}
    $Re(if) = - Im(f), \ Im(i f) = Re(f)$
    
    $\int_E{i f d \mu = i \int_E {f d \mu}}$
\end{proof}

\begin{remark}
    $\left|\int_E{f d \mu}\right| \leq \int_E{|f| d \mu}$
\end{remark}
\begin{proof}
    $\left| \int_E{f d \mu} \right| = e^{i \alpha} \cdot \int_E{f d \mu} = \int_E{e^{i \alpha} f d \mu} = \\ = \int_E{Re(e^{i \alpha} f) d \mu} + i \cdot \underbrace{\int_E{Im(e^{i \alpha}f) d \mu}}_{= 0 \text{, т.к. слева от равенства вещ. число}} = \int_E{Re(e^{i \alpha}f) d \mu} \leq \int_E{\left| Re(e^{i \alpha} f) d \mu \right|} \leq \int_E{\left| e^{i \alpha}\right|  d \mu } = \int_E{|f| d \mu}$.

    $|Re(f)|, |Im(f)| \leq |f|$
    
    $|f| \leq |Re(f)| + |Im(f)|$
\end{proof}

\begin{theorem}
    (О счетной аддитивности интеграла).

    Пусть $f \geq 0$ -- измеримая и $E = \bigsqcup_{n=1}^{\infty} E_n$.
    
    Тогда $\int_E{f d \mu} = \sum_{n=1}^{\infty} \int_{E_n}{f d \mu}$
\end{theorem}
\begin{proof}
    $\sum_{n=1}^{\infty} \int_{E_n}{f d \mu} = \lim{\sum_{k=1}^{n} \int_{E_k}{f d \mu}} = \lim{\int_{\bigsqcup_{k=1}^{n} E_k}{f d \mu}} = \lim{\int_E\left({\underbrace{\mathds{1}_{\bigsqcup_{k=1}^{n} E_k} f}_{:= g_n} d \mu}\right)} = \lim{\int_E{g_n d \mu}} \underbrace{=}_{\text{т. Леви}} \int_E{f d \mu}$

    $0 \leq g_1 \leq g_2 \leq \dots , \ \lim{g_n} = f, \ g_n(x) = f(x)$ если $x \in \bigsqcup_{k=1}^{n}E_k$.
\end{proof}

\begin{consequence}
    \begin{enumerate}
        \item Если $f \geq 0$ -- измеримая, то $\nu E := \int_E{f d \mu}$ -- мера, заданная на той же $\sigma$-алгебре, что и $\mu$.
        \item Если $f \geq 0$ и $E_1 \subset E_2 \subset \dots, \ E = \bigcup_{n=1}^{\infty}E_n$, то $\int_E{f d \mu} = \lim{\int_{E_n}{f d \mu}}$
        \item Если $f$ -- суммируема и $E_1 \supset E_2 \supset \dots, \ E = \bigcap_{n=1}^{\infty} E_n$, то $\int_E{f d \mu} = \lim{\int_{E_n}{f d \mu}}$
        \item Если $f$ -- суммируема на $E, \ \epsilon > 0$, то $\exists A \subset E:\ \mu A < +\infty \land \int_{E \setminus A}{|f| d \mu} < \epsilon$
    \end{enumerate}
\end{consequence}

\begin{proof}
    \begin{enumerate}
        \item $\nu \emptyset = \int_{\emptyset}{f d \mu} = 0$ + счетная аддитивность из теоремы: $\int_{E}{f_{\pm} d \mu} = \sum_{n=1}^{\infty} \int_{E_n}{f_{\pm} d \mu}$ все конечно, поэтому можно вычитать.
        \item {
            $\nu A := \int_A{f d \mu}$ -- мера $\implies \nu A$ непрерывна снизу.

            $\underbrace{\nu E}_{\int_E{f d \mu}} = \underbrace{\lim {\nu E_n}}_{\lim{\int_{E_n}{f d \mu}}}$
        }
        \item {
            $\nu_{\pm} A := \int_{A}{f_{\pm} d \mu}$, $\nu_{\pm} A$ -- конечные меры $\implies \nu_{\pm}$ -- непрерывна сверху.

            $\implies \int_E{f_{\pm} d \mu} = \nu_{\pm}E = \lim \nu_{\pm} E_n = \lim{\int_{E_n}{f_{\pm} d \mu}}$
        }
        \item {
            $E_n := E \{ |f| \leq \frac{1}{n} \} \implies E_n \supset E_{n+1}$

            $\bigcap_{n=1}^{\infty}E_n = E\{ f = 0 \} \implies \lim{\int_{E_n}{|f| d \mu}} = \int_{E\{ f = 0 \}}{|f| d \mu} = 0 \implies \exists n: \ \epsilon > \int_{E_n}{|f| d \mu} \geq \left| \int_{E_n}{f d \mu} \right|$

            $A := E \setminus E_n = E \{ |f| > \frac{1}{n} \}$

            $\mu A \underbrace{\leq}_{\text{Чебышев}} \frac{\int_E{|f|d \mu}}{\frac{1}{n}} < +\infty$
        }
    \end{enumerate}
\end{proof}

\begin{theorem}
    (Абсолютная непрерывность интеграла).

    $f$ -- суммируема на $E$, тогда $\forall \epsilon: \ \exists \delta > 0$, т.ч. $\forall e $ -- измер. $\mu e < \delta \implies | \int_e f d \mu | < \epsilon$
\end{theorem}
\begin{proof}
    $\int_E |f| d \mu < +\infty \implies \exists \underbrace{\phi}_{\leq f}$ -- неотрицательная простая, т.ч. $\\ \int_E |f| d \mu < \int_E \phi d \mu + \epsilon$.

    Пусть $C$ -- наибольшее значение $\phi$. Возьмем $\delta = \frac{\epsilon}{C}$.

    Если $\mu e < \delta$, то $\int_e |f| d \mu < \underbrace{\int_e \phi d \mu + \epsilon}_{\leq \int_e C d \mu + \epsilon \leq \epsilon + \epsilon}$ -- это следует из того, что $|f| - \phi \geq 0, \\ \int_{e} (|f| - \phi) d \mu \leq \int_E (|f| - \phi) d \mu < \epsilon$.
\end{proof}

\begin{consequence}
    Если $f$ суммируема на $E$ и $\mu A_n \rightarrow 0, \ A_n \subset E$, то $\int_{A_n} f d \mu \rightarrow 0$.
\end{consequence}
\begin{proof}
    Берем $\epsilon > 0$ и $\delta > 0$ для него из теоремы, тогда если $\mu A_n < \delta$, то $\\ |\int_{A_n} f d \mu| < \epsilon$
\end{proof}

\begin{definition}
    Пусть $\mu$ и $\nu$ меры на одной $\sigma$-алгебре $\mathcal{A}$. Если существует измеримая функция $w \geq 0$, т.ч. $\forall A \in \mathcal{A}, \ \nu A = \int_{A} w d \mu$.
    
    Тогда $w$ плотность меры $\nu$ относительно меры $\mu$.
\end{definition}

\begin{remark}
    Если $w$ существует, то $\nu$ обладает свойством: если $\mu e = 0$, то $\nu e = 0$.
\end{remark}

\begin{theorem}
    Пусть $f, g$ -- суммируемые функции. Если $\forall A$ -- измерим. $\int_A f d \mu = \int_A g d \mu$,
    то $f = g$ почти везде.
\end{theorem}
\begin{proof}
    $h := f - g, \ E_+ := E \{ f \geq g \}, \ E_- := E \{ f < g \}$

    $\int_E |h| d \mu = \underbrace{\int_{E_+} h d \mu}_{=0} - \underbrace{\int_{E_-} h d \mu}_{=0} = 0 \implies h = 0$ почти везде.
\end{proof}

\begin{theorem}
    (Единственность плотности).

    Если $\nu$ -- $\sigma$-конечная мера (на $\sigma$-алгебре $\mathcal{A}$) и $w$ -- плотность $\nu$ относительно $\mu$, то $w$ -- единственна с точностью до \textbf{почти везде}.
\end{theorem}
\begin{proof}
    Так как наша мера -- $\sigma$-конечна, то все пространство представляется как $X = \bigsqcup_{n=1}^{\infty}X_n$, т.ч. $\nu X_n < +\infty \implies $ т.к. $w$ -- плотность $\nu|_{X_n}$ относительно $\mu|_{X_n} \implies w$ -- суммируема на $X_n$.

    Пусть $w_1, w_2$ -- плотности $\nu$ относительно $\mu$ на сужении одного кусочка, тогда по определению плотности верно, что $\forall A \in \mathcal{A}: \nu A = \int_{A} {w_1 d \mu} = \int_{A} {w_2 d \mu} \underbrace{\implies}_{\text{по пред. теореме}} w_1 = w_2$ почти везде.
    
    Ну если две плотности на каждом из кусочков отличаются на множество нулевой меры, тогда и на объединении кусочков тоже будут отличаться на множество нулевой меры, тогда плотность единственна почти везде и на всей $\sigma$-алгебре.
\end{proof}

\begin{definition}
    $\nu, \mu$ -- меры, заданные на одной $\sigma$-алгебре. $\nu$ абсолютно непрерывна относительно $\mu$, если $\forall e$ -- измер., т.ч. $\mu e = 0 \implies \nu e = 0$.

    Обозначение $\nu \prec \mu$ или $\nu \ll \mu$.
\end{definition}


\begin{theorem}
    (\textbf{Радона-Никодима}).

    Пусть меры $\mu$ и $\nu$ заданы на одной $\sigma$-алгебре. Тогда $\nu \prec \mu \Leftrightarrow $ существует плотность меры $\nu$ относительно $\mu$.
\end{theorem}

\begin{theorem}
    $w$ --  плотность $\nu$ относительно $\mu$. Тогда 

    \begin{enumerate}
        \item Если $f \geq 0$, то $\int_E f d \nu = \int_E f w d \mu \ : (*)$
        \item $fw$ -- суммируема, относительно $\mu$ $\Leftrightarrow$ f -- суммируема относительно $\nu$, и в этом случае есть формула $(*)$
    \end{enumerate}
\end{theorem}
\begin{proof}
    \begin{enumerate}
        \item {
            Пусть $f = \mathds{1}_A$, тогда $\int_E f d \nu = \nu (A \cap E) = \int_{A \cap E} w d \mu = \int_E \mathds{1}_A w d \mu$. По линейности $(*)$ верна для неотрицательный простых.

            Пусть $f \geq 0$ -- измер. Тогда найдутся простые $0 \leq \phi_1 \leq \phi_2 \leq \dots$ ($0 \leq w \phi_1 \leq w \phi_2 \leq \dots$) и $\phi_n \rightarrow f$ поточечно. $\underbrace{\int_E \phi_n d \nu}_{\rightarrow \int_E f d \nu} = \underbrace{\int_E \phi_n w d \mu}_{\rightarrow \int_E f w d \mu}$ -- по т. Леви.
        }
        \item {
            $\int_E |f| d \nu = \int_E |f| w d \mu \implies f$ -- суммируема относительно $\nu$ $\Leftrightarrow f w$ суммируема относительно $\mu$

            $\int_E f_{\pm} d \nu = \int_E f_{\pm} w d \mu$ и вычитаем.
        }
    \end{enumerate}
\end{proof}
\begin{properties}
    Неравенство Гельдера.

    Пусть $p, q > 1$ и $\frac{1}{p} + \frac{1}{q} = 1$. Тогда $\int_E |fg|d \mu \leq \left(\int_E |f|^p d \mu\right)^{\frac{1}{p}} \cdot \left(\int_E |g|^q d \mu\right)^{\frac{1}{q}} = A \cdot B$
\end{properties}
\begin{proof}
    Пусть $f, g \geq 0$ (просто чтобы не писать модули), $A^p := \int_E f^p d \mu, \ B^q := \int_E g^q d \mu$.

    Случай $A = 0. \implies$ $f^p = 0$ почти везде $\implies f = 0$ почти везде $\implies f g  = 0$ почти везде $\implies \int_E f g d \mu = 0$.

    Можно считать, что $A, B > 0$.

    Случай $A = +\infty$. Очевидно.

    Можно считать $0 < A , B < +\infty$.
    
    $u := \frac{f}{A}, \ v:= \frac{g}{B}$

    $\int_E u^p d \mu = 1 = \int_E v^q d \mu, \ uv \leq \frac{u^p}{p} + \frac{v^q}{q}$ верно (Упражнение, ну конечно. Фиксируем одну из переменных как параметр и исследуем нер-во по второй переменной).

    Интегрируем полученное нер-во: $\frac{1}{AB} \int_E f g d \mu = \int_E u v d \mu \leq \frac{1}{p} \underbrace{\int_E u^p d \mu}_{=1} + \frac{1}{q} \underbrace{\int_E v^q d \mu}_{=1} = \frac{1}{p} + \frac{1}{q} = 1$ 
\end{proof}

\begin{properties}
    Неравенство Минковского.

    $p \geq 1$, тогда $\left(\int_E |f + g|^p d \mu\right)^{\frac{1}{p}} \leq \left( |f|^p d \mu \right)^{\frac{1}{p}} + \left( |g|^p d \mu \right)^{\frac{1}{p}}$
\end{properties}
\begin{proof}
    Можно считать, что $f, g \geq 0$, также можно считать, что $\int_E f^p d \mu$ и $\int_E g^p d \mu < +\infty$.

    Проверим, что $\int_E (f + g)^p d \mu < +\infty$:
    
    $f + g \leq 2 \max \{ f, g \} \implies (f + g)^p \leq 2^p \max\{ f^p, g^p \} \leq 2^p (f^p + g^p)$

    $\underbrace{\int_E (f + g)^p d \mu}_{=: C^p} \leq 2^p \left(\int_E f^p d \mu + \int_E g^p d \mu \right) < +\infty$ -- показали, что левая часть конечна.

    Можем считать, что $0 < C < +\infty$:
    
    $C^p = \int_E (f+g)^p d \mu = \int_E (f + g) (f + g)^{p - 1} d \mu = \int_E f (f+g)^{p-1} d \mu + \int_E g (f + g)^{p - 1} d \mu$


    Пусть $\frac{1}{p} + \frac{1}{q} = 1, \ q = \frac{p}{p-1}, \ (p - 1) q = p$, тогда:

    $\int_E{f \cdot (f + g)^{p - 1} d \mu} \underbrace{\leq}_{\text{нер-во Гельдера}} \left(\int_E f^p d \mu\right)^{\frac{1}{p}} \cdot \left(\int_E ((f+g)^{p-1})^{q} d \mu\right)^{\frac{1}{q}} = \left(\int_E f^p d \mu\right)^{\frac{1}{p}} \cdot \underbrace{\left(C^p\right)^{\frac{1}{q}}}_{= C^{p - 1}} \leq \left( \int_E f^p d \mu \right)^{\frac{1}{p}} C^{p - 1} + \left(\int_E g^p d \mu\right)^{\frac{1}{p}} \cdot C^{p - 1}$ -- сокращаем на $C^{p - 1}$.
\end{proof}

\Subsection{Предельный переход под знаком интеграла}
\begin{theorem}
    \textbf{Леви}.

    $0 \leq f_1 \leq f_2 \leq \dots$ и $f = \lim{f_n}$, тогда $\lim{\int_E{f_n d \mu}} = \int_E f d \mu$.
\end{theorem}
\begin{consequence}
    Пусть $u_n \geq 0.$ Тогда $\int_E {\sum_{n=1}^{\infty} u_n d \mu} = \sum_{n=1}^{\infty} \int_E {u_n d \mu}$
\end{consequence}
\begin{proof}
    $s_n := \sum_{k=1}^{n} u_k, \ 0 \leq s_1 \leq s_2 \leq \dots$ и $s_n \rightarrow s := \sum_{n=1}^{\infty} u_n$.

    $\int_E{s d \mu} = \lim{\int_E{s_n d \mu}} = \lim{\int_E{\sum_{k=1}^{n} u_k d \mu}} = \lim{\sum_{k=1}^{n} \int_E{u_k d \mu}} = \sum_{k=1}^{\infty} \int_E{u_k d \mu}$
\end{proof}

\begin{consequence}
    Если $\sum_{n=1}^{\infty} \int_E{|f_n| d \mu} < +\infty$, то $\sum_{n=1}^{\infty} f_n(x)$ сходится при почти всех $x \in E$.
\end{consequence}
\begin{proof}
    $+\infty > \sum_{n=1}^{\infty} \int_E{|f_n| d \mu} = \int_E{\sum_{n=1}^{\infty} |f_n| d \mu} \implies \sum_{n=1}^{\infty} |f_n|$ -- суммир.

    $\implies \sum_{n=1}^{\infty} |f_n|$ почти везде конечна $\implies \sum_{n=1}^{\infty} f_n(x)$ абс. сходится при почти всех $x \in E \implies$ сходится при почти всех $x \in E$.
\end{proof}

\begin{lemma}
    \textbf{Фату}.

    Если $f_n \geq 0$, то $\int_E{\underline{\lim}{f_n d \mu}} \leq \underline{\lim}{\int_E{f_n d \mu}}$.
\end{lemma}
\begin{proof}
    $\underline{\lim}{f_n} = \lim{\underbrace{\inf\{ f_n, f_{n+1}, \dots \}}_{=: g_n}}$

    $0 \leq g_1 \leq g_2 \leq \dots$ и $g_n \rightarrow \underline{\lim}{f_n}$

    $\underbrace{\implies}_{\text{теорема Леви}} \underbrace{\lim_{\int_E{g_n d \mu}}}_{= \underline{\lim}{\int_E{g_n d \mu}} \leq \underline{\lim}{\int_E{f_n d \mu}}} = \int_E{\underline{\lim}{f_n d \mu}}$

    $g_n \leq f_n \implies \int_E{g_n d \mu} \leq \int_E{f_n d \mu} \implies \underline{\lim}{\int_E{g_n d \mu}} \leq \underline{\lim}{\int_E{f_n d \mu}}$
\end{proof}

\begin{remark}
    Равенства может и не быть:

    $\mu = \lambda, \ E = \mathbb{R}, \ f_n = \mathds{1}_{[n, +\infty)}$

    $\int_E{f_n d \mu} = +\infty$, но $f_n \rightarrow 0$

    Из этих двух условие следует, что $\int_E{\underline{\lim}{f_n d \mu}} = \int_E{0 d \mu} = 0$
\end{remark}

% todo: put all in-integrals expressions in {}, aka \int_E ... -> \int_E {...}


\begin{consequence}
    (Усиленный вариант теоремы Леви).

    Пусть $0 \leq f_n \leq f$ и $f = \lim{f_n}$. Тогда $\lim{\int_E{f_n d \mu}} = \int_E{f d \mu}$
\end{consequence}
\begin{proof}
    $f_n \leq f \implies \int_E{f_n d \mu} \leq \int_E{f d \mu} \implies \int_E{f d \mu} = \int_E{\underline{\lim}{f_n} d \mu} \leq \underline{\lim}{\int_E{f_n d \mu}} \leq \overline{\lim}{\int_E{f_n d \mu}} \leq \int_E{f d \mu}$

    $\implies \underline{\lim} = \overline{\lim} = \int_E{f d \mu} \implies \lim{\int_E{f_n d \mu}} = \int_E{f d \mu}$
\end{proof}

\begin{theorem}
    Лебега о предельном переходе (о мажорируемой сходимости).

    Пусть $f = \lim{f_n}$ и $|f_n| \leq \underbrace{F}_{\text{суммируемая мажоранта}}$ -- суммируема на $E$.

    Тогда $\lim{\int_E{f_n d \mu}} = \int_E{f d \mu}$, более того $\lim{\int_E{|f_n - f|d \mu}} = 0$
\end{theorem}
\begin{proof}
    $g_n := 2F - |f_n - f| \leq 2F$ и $g_n \rightarrow 2F$.
    
    $g_n \geq 2F - |f_n| - |f| \geq 0$.

    Тогда предел $\lim{\int_E{g_n d \mu}} = 2 \int_E{F d \mu}$

    $\int_E{g_n d \mu} = \int_E{2 F d \mu} - \int_E{|f_n - f| d \mu}$

    Из двух строчек выше делаем вывод, что $\underbrace{\int_E{|f_n - f| d \mu}}_{\geq \ \left| \int_E{(f_n - f) d \mu} \right| \ = \ \left| \int_E{f_n d \mu} - \int_E{f d \mu} \right|} \rightarrow 0$
\end{proof}
\begin{remark}
    \begin{enumerate}
        \item {
            Без суммир. мажоранты неверно:

            \begin{equation}
                f_n = n \cdot \mathds{1}_{[0, \frac{1}{n}]} \rightarrow f =
                \begin{cases}
                    +\infty, \text{  в точке 0} \\
                    0, otherwise \\
                \end{cases}
            \end{equation}
            

            $\int_{[0, 1]}{f d \lambda} = 0, \ \int_{[0, 1]}{f_n d \lambda} = 1, \ F:=\sup{f_n}, \ F(x) = n \text{ при } \frac{1}{n+1} < x \leq \frac{1}{n}$
        }
        \item {
            Поточечную сходимость можно заменить на сходимость почти везде, можно заменить и на сходимость по мере.
        }
    \end{enumerate}
\end{remark}

\begin{theorem}
    Пусть $f \in C[a, b]$. Тогда $\int_a^b {f} = \int_{[a, b]}{f d \lambda}$.
\end{theorem}
\begin{proof}
    % todo: picture
    $a = x_0$

    $b = x_n$

    $S_* := \sum_{k=1}^{n} \min_{t \in [x_{k-1}, \ x_k]} {f(t) \cdot (x_k - x_{k-1})}$

    $S^* := \sum_{k=1}^{n} \max_{t \in [x_{k-1}, \ x_k]} {f(t) \cdot (x_k - x_{k-1})}$

    % todo: second picture

    Если мелкость дробления $\rightarrow 0$, то $S_*, S^* \rightarrow \int_a^b {f}$.

    $g_*(x) := \min_{t \in [x_{k - 1}, \ x_k]}{f(t)}$ при $x \in [x_{k - 1}, \ x_k]$

    $g^*(x) := \max_{t \in [x_{k - 1}, \ x_k]}{f(t)}$ при $x \in [x_{k - 1}, \ x_k]$

    $\int_{[a, b]}{g_* d \lambda} = S_*, \ \int_{[a, b]}{g^* d \lambda} = S^*$

    $g_* \leq f \leq g^*$ почти везде.

    $\underbrace{S_*}_{\rightarrow \ \int_a^b f} = \int_{[a, b]}{g_* d \lambda} \leq \int_{[a, b]}{f d \lambda} \leq \int_{[a, b]}{g^* d \lambda} = \underbrace{S^*}_{\rightarrow \ \int_a^b{f}} \implies \int_{[a, b]}{f d \lambda} = \int_a^b {f}$
\end{proof}
\begin{remark}
    На самом деле это верно для любой функции, интегрир. по Риману на $[a, b]$.
\end{remark}

\begin{theorem}
    (Критерий Лебега интегрированности по Риману).

    $f: [a, b] \rightarrow \mathbb{R}$, тогда $f$ -- интегрируема по Риману $\Leftrightarrow$ множество точек разрыва $f$ имеет нулевую меру Лебега.
\end{theorem}
\begin{example}
    Возьмем $f: [0, 1] \rightarrow \mathbb{R}, \ f = \mathds{1}_{[0, 1] \cap \mathbb{Q}}$.

    $f = 0$ почти везде $\implies \int_{[0, 1]}{f d \lambda} = 0$, но точки разрыва -- весь отрезок $[0, 1]$.
\end{example}

\Subsection{Произведение мер}
\begin{definition}
    $\left(X, \mathcal{A}, \mu\right)$ и $\left(Y, \mathcal{B}, \nu\right)$ -- простарнства с $\sigma$-конечными мерами.

    $\mathcal{P} = \{ A \times B : \ A \in \mathcal{A}, \ B \in \mathcal{B}, \ \mu A < +\infty \  \land \ \nu B < +\infty \}$

    $m_0 (A \times B) = \mu A \cdot \nu B < +\infty, \ A \times B$ -- измеримый прямоугольник.
\end{definition}
\begin{theorem}
    $\mathcal{P}$ -- полукольцо, а $m_0$ -- $\sigma$-конечная мера на нем.
\end{theorem}
\begin{proof}
    $\{ A \in \mathcal{A}: \ \mu A < +\infty \}$ и $\{ B \in \mathcal{B}: \ \nu B < +\infty \}$ -- полукольца (проверяем определение полукольца для обоих множеств).

    $\mathcal{P}$ -- декартово произведение полуколец, то есть тоже полукольцо (эта по теореме, которая была выше).

    Проверяем, что $m_0$ -- мера. Пусть $A \times B = \bigsqcup_{k=1}^{\infty} A_k \times B_k$.

    $\mathds{1}_A(x) \times \mathds{1}_B(y) = \mathds{1}_{A \times B} (x, y) = \sum_{k=1}^{\infty} \mathds{1}_{A_k \times B_k} (x, y) = \sum_{k=1}^{\infty} \mathds{1}_{A_k}(x) \times \mathds{1}_{B_k}(y)$

    $\int_Y {\mathds{1}_A(x) \cdot \mathds{1}_B(Y) d \nu (y)} = \sum_{k=1}^{\infty} \int_Y{\mathds{1}_{A_k}(x) \cdot \mathds{1}_{B_k}(y) d \nu (y)} = \sum_{k=1}^{\infty} \mathds{1}_{A_k} (x) \cdot \nu B_k$

    $\int_X {\mathds{1}_A(x) \nu B d \mu(x)} = \sum_{k=1}^{\infty} \int_{X} {\mathds{1}_{A_k}(x) \cdot \nu B_k d \mu (x)} = \sum_{k=1}^{\infty} \mu A_k \cdot \nu B_k = \sum_{k=1}^{\infty} m_0 (A_k \times B_k)$

    $\sigma$-конечность $m_0$: $X = \bigsqcup_{j=1}^{\infty} X_j$, $Y = \bigsqcup_{j=1}^{\infty} Y_j$, $\mu X_j < +\infty, \ \nu Y_k < +\infty$

    $X \times Y = \bigsqcup_{k, j = 1}^{\infty} X_j \times Y_k$

    $m_0 (X_j \times Y_k) < +\infty$.
\end{proof}

\begin{definition}
    $(X, \mathcal{A}, \mu)$ и $(Y, \mathcal{B}, \nu)$ -- пространства с $\sigma$-конечными мерами. Произведения мер $\mu$ и $\nu$ -- стандратное продолжение меры $m_0$.

    Обозначение: $\mu \times \nu, \ \mathcal{A} \otimes \mathcal{B}$ -- $\sigma$-алгебра, на которую продолжили. $(X \times Y, \mathcal{A} \otimes \mathcal{B}, \mu \times \nu)$
\end{definition}

\begin{properties}
    \begin{enumerate}
        \item Декартово произвдедение измер мн-в -- измеримо.
        \item Если $\mu e = 0$, то $(\mu \times \nu)(e \times Y) = 0$.
    \end{enumerate}
\end{properties}
\begin{proof}
    \begin{enumerate}
        \item {
            $A \in \mathcal{A} \implies A = \bigcup_{n=1}^{\infty} A_n, \ \mu A_n < +\infty$

            $B \in \mathcal{B} \implies B = \bigcup_{n=1}^{\infty} B_n, \ \nu B_n < +\infty$

            $A \times B = \bigcup_{k, n = 1}^{\infty} \underbrace{A_k \times B_k}_{\in \mathcal{P}}$ -- измер.
        }
        \item {
            $Y = \bigsqcup_{k=1}^{\infty} Y_k, \ \nu Y_k < +\infty$

            $e \times Y = \bigsqcup_{k=1}^{\infty} e \times Y_k, \ (\mu \times \nu) (e \times Y_k) = \mu e \cdot \nu Y_k = 0$
        }
    \end{enumerate}
\end{proof}

\begin{remark}
    Обозначения: $C \subset X \times Y, \ x \in X$.

    $C_x := \{ y \in Y: (x, y) \in C \}$ -- сечения мн-ва $C$.

    $C^y := \{ x \in X: (x, y) \in C \}$

    %todo: picture
\end{remark} 
\begin{consequence}
    \begin{enumerate}
        \item {
            $\left(\bigcup_{\alpha \in I} C_{\alpha}\right)_x = \bigcup_{\alpha \in I} (C_{\alpha})_x$
        }
        \item {
            $\left( \bigcap_{\alpha \in I} C_{\alpha} \right)_x = \bigcap_{\alpha \in I} (C_{\alpha})_x$
        }
    \end{enumerate}
\end{consequence}

\begin{definition}
    Пусть функция $f$ задана на мн-ве $E$, за исключением некоторого мн-ва $e$, $\mu e  = 0$. Если $f$ измерима на $E \setminus e$, то $f$ измерима на $E$ в \textbf{широком смысле}.
\end{definition}

\begin{definition}
    Система множеств  -- \textbf{монотонный класс}, если 

    \begin{enumerate}
        \item $E_1 \subset E_2 \subset E_3 \subset \dots, \ E_n \in \epsilon \implies \bigcup_{n=1}^{\infty} E_n \in \epsilon$
        \item $E_1 \supset E_2 \supset E_3 \supset \dots, \ E_n \in \epsilon \implies \bigcap_{n=1}^{\infty} E_n \in \epsilon$
    \end{enumerate}
\end{definition}

\begin{theorem}
    Если монотонный класс содержит алгебру $\mathcal{A}$, то он содержит и $\mathcal{B} (\mathcal{A})$.
\end{theorem}
\begin{proof}
    Докажем, что минимальный монотонный класс $\mathcal{M}$, содержащий $\mathcal{A}$ -- $\sigma$-алгебра.

    Рассмотрим $A \in \mathcal{A}, \ \mathcal{M}_{A} := \{ B \in \mathcal{M} : \ A \cap B \in \mathcal{M} \ \land \ A \cap (X \setminus B) \in \mathcal{M} \}$ -- монотонный класс, содержащий $\mathcal{A}$.

    Если $B \in \mathcal{A}$, то $B \cap A \in \mathcal{A} \subset \mathcal{M}$ и $A \cap (X \setminus B) \in \mathcal{A} \subset \mathcal{M} \implies \mathcal{M}_A \supset \mathcal{A}$

    $E_1 \subset E_2 \subset \dots, \ E_n \in \mathcal{M}_A \implies E_n \cap A \in \mathcal{M} \implies \bigcup_{E_n} \cap A = \bigcup(E_n \cap A) \in \mathcal{M}$

    Следовательно $\mathcal{M}_{A} = \mathcal{M} \implies \forall B \in \mathcal{M}, \ A \cap B \in \mathcal{M} \ \land \ A \setminus B \in \mathcal{M}$

    $\implies \mathcal{M}$ -- симметричная структура.
    
    Рассмотрим $B \in \mathcal{M}$: $\mathcal{N}_B := \{ C \in \mathcal{M}: \ B \cap C \in \mathcal{M} \}$ -- монотонный класс, содержащий $\mathcal{A}$ (проверка по аналогии с предыдщуим случаем).

    $\implies \mathcal{N}_B = \mathcal{M} \implies \forall C \in \mathcal{M}, \ B \cap C \in \mathcal{M} \implies \mathcal{M}$ -- алгебра. 

    $A = \bigcup_{n=1}^{\infty} A_n, \ E_n = \bigcup_{k=1}^{n} A_k \in \mathcal{M}, \ E_1 \subset E_2 \subset \dots$

    $\implies \underbrace{\bigcup E_n}_{= A} \in \mathcal{M}$, так как $\mathcal{M}$ -- монотонный класс.
\end{proof}

\begin{theorem}
    \textbf{Принцип Кавальери}.

    $(X, \mathcal{A}, \mu), (Y, \mathcal{B}, \nu)$ - пространства с полными $\sigma$-конечными мерами.

    $C \in \mathcal{A} \otimes \mathcal{B}, \ m = \mu \times \nu$. Тогда
    \begin{enumerate}
        \item {
            $C_x \in \mathcal{B}$ при почти всех $x \in X$.
        }
        \item {
            $\phi(x) := \nu C_x$ измеримая в широком смысле.
            % todo: picture
        }
        \item {
            $m C = \int_X {\nu C_x d \mu(x)}$
            % todo: picture
        }

    \end{enumerate}
\end{theorem}
\begin{proof}
    Меры конечны и $C \in \underbrace{\mathscr{B}}_{\text{борелевская оболочка (\textcolor{blue}{см. определение \ref{borel-shell}})}}(\mathcal{A} \times \mathcal{B})$.

    $\mathcal{E}$ -- система мн-в, в $\mathscr{B}(\mathcal{A} \times \mathcal{B})$, такая что, если $E \in \mathcal{E}$, то $E_x \in \mathcal{B} \ \forall x \in X$ и $\phi(x) = \nu E_x$ -- измеримая функция.

    \textbf{Шаг 1}. $\mathcal{E} = \mathscr{B}(\mathcal{A} \times \mathcal{B})$

    \textbf{a}. $\mathcal{E}$ -- измеримая система.

    $(X \times Y \setminus E)_x = Y \setminus E_x \in \mathcal{B}, \ \nu (Y \setminus E_x) = \nu Y - \phi(x)$ -- измеримая.

    \textbf{б}. $E_1 \subset E_2 \subset E_3 \subset \dots$ из $\mathcal{E} \implies \bigcup E_n \in \mathcal{E}$.

    $\left( \bigcup_{n=1}^{\infty} E_n \right)_x = \bigcup_{n=1}^{\infty}\underbrace{\left( E_n \right)_x}_{\in \mathcal{B}}$

    $\nu \left( \bigcup_{n=1}^{\infty} (E_n)_x \right) = \lim{\nu (E_n)_x}$ -- измеримая функция.

    \textbf{в}. $E_1 \supset E_2 \supset E_3 \supset \dots $ из $\mathcal{E} \implies \bigcap_{n=1}^{\infty} E_n \in \mathcal{E}$ (можно переходить к дополнениям).

    \textbf{г}. (б) + (в) $\implies \mathcal{E}$ -- монотонный класс.

    \textbf{д}. $\mathcal{E} \supset $ измеримый прямоугольник $E = \mathcal{A} \times \mathcal{B} \implies E_x = $
    $\begin{cases}
        B \text{, если } x \in \mathcal{A} \\ 
        \emptyset \text{, иначе} 
    \end{cases}$,

    $\nu E_x = 
    \begin{cases}
        0 \\
        \nu \mathcal{B}
    \end{cases}$
    -- измеримая функция.

    \textbf{e}. Если $E$ и $\tilde{E} \in \mathcal{E}$, то $E \sqcup \tilde{E} \in \mathcal{E}$.

    $(E \sqcup \tilde{E})_x = \underbrace{E_x}_{\in \mathcal{B}} \sqcup \underbrace{\tilde{E}_x}_{\in \mathcal{B}} \in \mathcal{B}$

    $\nu \left( (E \sqcup \tilde{E})_x \right) = \nu E_x + \nu \tilde{E}_x$ -- сумма измеримых функций.

    \textbf{ж}. $\mathcal{E}$ содержит дизъюнктивное объединение всевозможных изм. прямоугольников $\implies \mathcal{E}$ содержит кольцо $\implies \mathcal{E}$ содержит алгебру $\underbrace{\implies}_{\text{по т. о монотонном классе}} \mathcal{E} \supset \mathscr{B}(\mathcal{A} \times \mathcal{B})$.

    \textit{Мы сейчас проверили, что если $C \in \mathscr{B}(\mathcal{A} \times \mathcal{B})$, то первые два пункта теоремы выполнены. Давайте для этой эе упрощенной ситуации проверять 3-ий пункт.}
    
    \textbf{Шаг 2}. Формула (3) для $C \in \mathscr{B}(\mathcal{A} \times \mathcal{B})$.

    Рассмотрим $\int_X{\nu E_x d \mu(x) =: \tilde{m} E}$ -- хотим сказать, что это мера на $\mathscr{B}(\mathcal{A} \times \mathcal{B})$.

    Пусть $E_n$ -- дизъюнктны $\implies \tilde{m} (\bigsqcup E_n) = \int_X{\nu \left( \bigsqcup (E_n)_x \right) d \mu(x)} = \int_X {\sum_{n=1}^{\infty} \nu (E_n)_x d \mu(x)} = \sum_{n=1}^{\infty} \int_X {\nu (E_n)_x d \mu(x)} = \sum_{n=1}^{\infty} \tilde{m} E_n$.

    $m = \tilde{m}$ на измеримых прямоугольниках $\implies$ они совпадают.
    Получили, что хотели.

    \textbf{Шаг 3}. $m C = 0, \ C \in \mathcal{A} \otimes \mathcal{B} \implies$ найдется $\tilde{C} \in \mathscr{B} (\mathcal{A} \times \mathcal{B})$, т.ч. $C \subset \tilde{C}$ и $m \tilde{C} = 0$.

    $0 = m \tilde{C} = \int_X {\nu \tilde{C}_x d \mu (x)} \implies \nu \tilde{C}_x = 0$ при почти всех $x \in X$.

    $C_x \subset \tilde{C}_x \implies C_x \in \mathcal{B}$ при почти всех $x \in X$ и $\nu C_x = 0$ при потчи всех $x \in X$.

    $m C = 0 = \int_X {\nu C_x d \mu(x)}$.

    \textbf{Шаг 4}. $C \in \mathcal{A} \otimes \mathcal{B} \implies C = \tilde{C} \sqcup e, \ \tilde{C} \in \mathscr{B} (\mathcal{A} \times \mathcal{B}), \ me = 0$.

    $C_x = \underbrace{\tilde{C}_x}_{\text{изм. } \forall x \in X} \sqcup \underbrace{e_x}_{\text{изм. при почти всех } x}$, $\nu C_x = \nu \tilde{C}_x + \nu e_x = \nu \tilde{C}_x$.

    $m C = m \tilde{C} + m e = m \tilde{C} = \int_X {\nu \tilde{C}_x d \mu (x)} = \int_X {\nu C_x d \mu(x)}$.

    \textbf{Шаг 5}. $X = \bigsqcup_{n=1}^{\infty} X_n$, $Y = \bigsqcup_{k=1}^{\infty} Y_k$, $\mu X_n < +\infty$.

    $X \times Y = \bigsqcup_{n, k = 1}^{\infty} X_n \times Y_k$

    $C \in \mathcal{A} \otimes \mathcal{B}$, $C_{nk} = C \cap X_n \times Y_k \implies C_{nk}$ удовлетворяет теореме.

    $C_x = \bigsqcup_{n, k = 1}^{\infty} (C_{nk})_x$

    % todo: picture

    $m C = \sum_{n, k = 1}^{\infty} m C_{nk} = \sum_{n, k = 1}^{\infty} \int_{X} {\nu (C_{nk})_x d \mu (x)} = \int \sum \dots = \int_X {\nu C_x d \mu}$.
\end{proof}

\begin{remark}
    \begin{enumerate}
        \item Нужна лишь полнота $\nu$.
        \item {
            Измеримость всех $C_x$ не гарантирует измеримость $C$.

            \begin{proof}
                $\mathbb{R}^2, \ E \subset \mathbb{R}$ -- неизмеримое, $E \times [0, 1]$
            \end{proof}
        }
        \item {
            Среди $C_x$ могут попадаться неизмеримые.

            \begin{proof}
                $\mathbb{R}^2, \ E \subset \mathbb{R}$ -- неизмеримые, $\{0\} \times E$
            \end{proof}
        }
        \item {
            % todo: picture
            Хочется интегрировать не по $X$, а по проекции, то есть $P := \{ x \in X: \ C_x \not = \emptyset \}$. Но $P$ может быть неизмеримо.

            \begin{proof}
                % todo: picture

                $E \subset \mathbb{R}$ -- неизмеримое, решение проблемы, это взять $\tilde{P} := \{ x \in X: \ \nu C_x > 0 \}$ -- измеримое.
            \end{proof}
        }
    \end{enumerate}
\end{remark}

\begin{definition}
    $(X, \mathcal{A}, \mu)$ -- пр-во с $\sigma$-конечной мерой.

    $f: X \rightarrow \overline{\mathbb{R}}, \ f \geq 0, \ E \in \mathcal{A}$, $m = \mu \times \underbrace{\lambda_1}_{\text{одномерная мера Лебега}}$.

    График функции над мн-вом $E$:
    
    $\Gamma_f(E) := \{ (x, y) \in E \times \mathbb{R} : y = f(x) \}$
    
    Подграфик функции над мн-вом $E$:

    $\mathcal{P}_f(E) := \{ (x, y) \in E \times \mathbb{R}: 0 \leq y \leq f(x) \}$
\end{definition}

\begin{lemma}
    (Лемма 1).
    
    Если $f$ -- измеримая, то $m \Gamma_f = 0$.
\end{lemma}
\begin{proof}
    Пусть $\mu X < +\infty$. Возьмем $\epsilon > 0$ и $A_n := X \{ \epsilon \cdot n \leq f < \epsilon \cdot (n + 1) \}$

    $\Gamma_f \subset \bigsqcup_{n \in \mathbb{Z}} \left(A_n \times [\epsilon \cdot n, \epsilon \cdot (n + 1)]\right) =: A$.

    % todo: picture

    $m A = \sum_{n \in \mathbb{Z}} m \left( A_n \times [\epsilon \cdot n, \epsilon \cdot (n + 1)] \right) = \epsilon \cdot \sum_{n \in \mathbb{Z}} \mu A_n = \epsilon \cdot \mu X$ -- сколь угодно маленькое.

    Пусть $\mu$ -- $\sigma$-конечна. $X = \bigsqcup_{n=1}^{\infty} X_n, \ \mu X_n < +\infty$,
    
    $\Gamma_f = \bigsqcup_{n=1}^{\infty} \Gamma_f (X_n)$ -- нулевой меры.
\end{proof}

\begin{lemma}
    (Лемма 2).

    $f \geq 0$ -- измерима в широком смысле $\implies \mathcal{P}_f$ -- измеримое мн-во.
\end{lemma}
\begin{proof}
    \begin{enumerate}
        \item {
            Пусть $f$ -- простая $\implies f = \sum_{k = 1}^{n} a_k \mathds{1}_{A_k} \implies \mathcal{P}_f = \bigsqcup_{k=1}^{n} A_k \times [0, a_k]$ -- измеримое.
        }

        \item {
            Пусть $f$ -- измеримая $\implies 0 \leq \phi_1 \leq \phi_2 \leq \dots \leq \phi_n \rightarrow f$ -- простые $\phi_i$, $\mathcal{P}_{\phi_n} \subset \mathcal{P}_f$.

            $\mathcal{P}_f \setminus \Gamma_f \subset \bigcup_{n=1}^{\infty} \mathcal{P}_{\phi_n} \subset \mathcal{P}_f$.

            Берем $x \in X$.
            
            Если
            \begin{enumerate}
                \item {
                    $f(x) = +\infty$, то $\phi_n(x) \rightarrow +\infty$, над точкой $x$, $[0, \phi_n(x)]$ их объединие будет луч.
                }
                \item {
                    $f(x) < +\infty$, то $\phi_n(x) \rightarrow f(x)$, $\bigcup [0, \phi_n(x)] \supset [0, f(x)]$
                }
            \end{enumerate}
        }
    \end{enumerate}
\end{proof}

\begin{theorem}
   (О мере подграфика).
   
   $(X, \mathcal{A}, \mu)$ -- пространство с $\sigma$-конечной мерой, $f \geq 0, \ f : X \rightarrow \overline{\mathbb{R}}, \ m = \mu \times \lambda_1$.

   Тогда $f$ -- измеримая в широком смыслке $\Leftrightarrow \mathcal{P}_f$ -- измер. и в этом случае $\int_X {f d \mu} = m \mathcal{P}_f$.
\end{theorem}
\begin{proof}
    "$\Rightarrow$": Лемма 2.

    "$\Leftarrow$": принцип Кавальери для $\mathcal{P}_f$:

    \begin{equation}
        (\mathcal{P}_f)_x =
        \begin{cases}
            [0, +\infty) \text{, при } f(x) = +\infty \\
            [0, f(x)) \text{, при } f(x) < +\infty
        \end{cases}
    \end{equation}

    $\phi(x) := \lambda_1 (\mathcal{P}_f)_x = \underbrace{f(x)}_{\text{измеримая в широком смысле}}$

    $m \mathcal{P}_f = \int_X {\underbrace{\lambda \left( (\mathcal{P}_f)_x \right)}_{= f(x)} d \mu (x)}$ -- получили, что хотели.
\end{proof}

\begin{theorem}
    \textbf{Тонелли}.

    $(X, \mathcal{A}, \mu), \ (Y, \mathcal{B}, \nu)$ -- пространства с полными $\sigma$-конечными мерами.

    $f: \ X \times Y \rightarrow \overline{\mathbb{R}} \geq 0$, измеримая, $m = \mu \times \nu$.

    Тогда: 
    \begin{enumerate}
        \item {
            $f_x(y) := f(x, y)$ -- измерима, относительно $\nu$ в широком смысле при почти всех $x \in X$.
        }
        \item {
            $\phi(x) := \int_Y { f(x, y) d \nu(y) }$ -- измерима относительно $\nu$.
        }
        \item {
            $\int_{X \times Y} {f d m} = \int_{X} {\phi d \mu} = \int_{X} {\left(\int_{Y} {f(x, y) d \nu(y)}\right) d \mu(x)}$
        }
    \end{enumerate}
\end{theorem}
\begin{proof}
    \begin{enumerate}
        \item {
            Пусть $f = \mathds{1}_C$ (характеристическая функция мн-ва $C$), тогда $f_x(y) = \mathds{1}_{C_x}(y)$.

            $\int_Y {f_x(y)d \nu(y)} = \int_Y {\mathds{1}_{C_x}(y) d \nu(y)} = \nu C_{x}$

            $\int_{X \times Y} {f d m} = \int_{X \times Y} {\mathds{1}_C d m} = m C = \int_X {\nu C_x d \mu(x)} = \int_X {\phi d \mu}$.
        }
        \item {
            Пусть $f \geq 0$ -- простая, тогда $f = \sum_{k=1}^{n} a_k \mathds{1}_{A_k}$
        }
        \item {
            Пусть $f \geq 0$ -- измеримая, тогда берем последовательность простых функций $0 \leq f_1 \leq f_2 \leq \dots$, $\lim{f_n} = f$.

            $(f_n)_x(y)$ -- измерим. при почти всех $x$.

            $(f_n)_x \nearrow f_x$ -- измерим. при почти всех $x$.

            $\phi_n(x) = \int_Y {f_n(x, y) d \nu (y)}$ -- измерим. и $0 \leq \phi_1 \leq \phi_2 \leq \dots$.

            $\lim{\phi_n(x)} = \int_Y{\lim{f_n (x, y)} d \nu(y)} = \int_Y{f(x, y) d \nu(y)} = \phi(x)$ -- измерим.

            $\int_{X \times Y}{f d m} \underbrace{\leftarrow}_{\text{т. Леви}} \int_{X \times Y} {f_n d m} = \int_X {\phi_n d \mu} \rightarrow \int_X{\phi d \mu}$.
        }
    \end{enumerate}
\end{proof}


\begin{theorem}
    \textbf{Фубини}.

    $(X, \mathcal{A}, \mu), \ (Y, \mathcal{B}, \nu)$ -- пространства с полными $\sigma$-конечными мерами.

    $f: \ X \times Y \rightarrow \overline{\mathbb{R}} \geq 0$, суммируема, $m = \mu \times \nu$.

    Тогда: 
    \begin{enumerate}
        \item {
            $f_x(y) := f(x, y)$ -- суммируема, относительно $\nu$ в широком смысле при почти всех $x \in X$.
        }
        \item {
            $\phi(x) := \int_Y { f(x, y) d \nu(y) }$ -- суммируема относительно $\nu$.
        }
        \item {
            $\int_{X \times Y} {f d m} = \int_{X} {\phi d \mu} = \int_{X} {\left(\int_{Y} {f(x, y) d \nu(y)}\right) d \mu(x)}$
        }
    \end{enumerate}
\end{theorem}
\begin{proof}
    $(*): \int_{X \times Y} {|f| d m} < +\infty$ -- следует из суммируемости $f$.
    
    $(*) \underbrace{=}_{\text{т. Тонелли}} = \int_{X} {\underbrace{\int_Y {|f(x, y)| d \nu(y)}}_{:= \alpha(x)} d \mu(x)}$

    $\implies \alpha(x) = \underbrace{\int_Y{|f(x, y)| d \nu(y)}}_{\implies f_x \text{ -- суммируема при почти всех } x \in X}$ -- конечна при почти всех $x \in X$.

    $\int_X{|\phi| d \mu} = \int_X{\left|\int_Y{f(x, y)d \nu(y)}\right| d \mu (x)} \leq \int_X{\int_Y{\left|f(x, y)\right|d \nu(y)} d \mu (x)} = \int_{X \times Y} {|f| d m} < +\infty$

    $\implies \phi$ -- суммируема.

    $\int_{X \times Y} {f_{\pm} d m} = \int_X { \left(\int_Y{f_{\pm}(x, y) d \nu(y)}\right) d \mu(x)}$ и вычтем $f = f_+ - f_-$.
\end{proof}

\begin{consequence}
    Если $f \geq 0$ и измеримая или  $f$ -- суммируемая, то 

    $\textbf{(**)}$: $\int_X{\left( \int_Y{f(x, y) d \nu(y)} \right) d \mu(x)} = \int_Y{\left( \int_X{f(x, y) d \mu(x) }\right) d \nu(y)}$.
\end{consequence}

\begin{consequence}
    $(X, \mathcal{A}, \mu), \ (Y, \mathcal{B}, \nu)$ -- пространства с полными $\sigma$-конечными мерами.

    $f: X \rightarrow \overline{\mathbb{R}}$ -- суммируема по $\mu$, $g: Y \rightarrow \overline{\mathbb{R}}$ -- суммируема по $\nu$. 

    Тогда $h(x, y) = f(x) \cdot g(y)$ суммируема по $m = \mu \times \nu$ и $\int_{X \times Y} {h d m} = \int_X {f d \mu} \cdot \int_Y {g d \nu}$.
\end{consequence}
\begin{proof}
    $\int_{X \times Y}{|h| d m} \underbrace{=}_{\text{т. Тонелли}} = \int_X{\left( \int_Y{|f(x)| |g(y)| d \nu(x)} \right) d \mu(x)} =$
    
    $= \int_X{ |f(x)| \cdot \int_Y {|g(y)| d \nu(y)} d \mu(x)} = \int_Y {|g| d \nu} \cdot \int_X {|f| d \mu} < +\infty \implies h$ -- суммируема.

    По Фубини пишем все без модулей.
\end{proof}

\begin{remark}
    \begin{enumerate}
        \item {
            Суммируемости $f_x(y) = f(x, y), \ f^y(x) = f(x, y), \ \phi(x) = \int_X{f_x d \nu}, \ \psi(y) = \int_X{f^y d \mu}$ не хватает для суммируемости $f$ по мере $m$.
        }
        \item {
            Без суммируемости $f$ по $m$ равенства $\textbf{(**)}$ может не быть.

            \begin{example}
                $\mathbb{R}^2, \ f(x, y) = \frac{x^2 - y^2}{(x^2 + y^2)^2}, \ g(x, y) = \frac{2xy}{(x^2 + y^2)^2}$

                Первообразные:

                1. $\int{f(x, y) d x} = -\frac{x}{x^2 + y^2}$

                2. $\int{g(x, y) d x} = -\frac{y}{x^2 + y^2}$

                Подставляем:

                1. $\int_{[-1, 1]}{f(x, y) d x} = -\frac{x}{x^2 + y^2}|^{x=1}_{x=-1} = \frac{-2}{y^2 + 1}$

                $\int_{[-1, 1]} {\int_{[-1, 1]} {f(x, y) dx}dy} = -2 \int_{[-1, 1]}{\frac{dy}{y^2 + 1}} = -2 \cdot \arctan(y) |^1_{-1} = -\pi$

                $\int_{[-1, 1]} {\int_{[-1, 1]} {f(x, y) dy}dx} = \pi$ -- не совпали из-за отсутствия суммируемости.

                2. $\int_{[-1, 1]}{g(x, y) d x} = -\frac{y}{x^2 + y^2}|^{x=1}_{x=-1} = 0$
            \end{example}
        }
    \end{enumerate}
\end{remark}

\begin{theorem}
    $(X, \mathcal{A}, \mu)$ -- пространство с $\sigma$-конечной мерой, $f: X \rightarrow \overline{\mathbb{R}}$ -- измерим.

    $\int_X{|f| d \mu} = \int_{0}^{+\infty} {\mu X \{ |f| \geq t \}}dt$ (в скобках записана функция распределения).
\end{theorem}
\begin{proof}
    $m = \mu \times \lambda_{1}$.

    $\int_X{|f| d \mu} = m \mathcal{P}_{|f|} = \int_{[0, +\infty]} {\left(\int_{X} {\underbrace{\mathds{1}_{\mathcal{P}_{|f|}}(x, t)}_{=1 \Leftrightarrow |f(x)| \geq t} d \mu(x)}\right) d \lambda_1(t)} = \int_{[0, +\infty]}{\mu X \{ |f| \geq t \} d \lambda_1(t)}$.

    % todo: picture
\end{proof}

\begin{consequence}
    \begin{enumerate}
        \item {
            В условии теоремы $\int_X {|f| d \mu} = \int_{0}^{+\infty} {\mu X\{ |f| > t \}dt}$

            \begin{proof}
                $g(t) := \mu X \{ |f| \geq t \}$ -- монотонно возраст., не более чем счтеное число точек разрыва.

                $\mu X\{ |f| > t \} = \lim{\mu X \{ |f| \geq t+\frac{1}{n} \}} = \lim_{n \rightarrow \infty}{g(t + \frac{1}{n})} = \lim_{s \rightarrow t+}{g(s)} = g(t)$ при почти всех $t$.

                $X \{ |f| > t \} = \bigcup_{n=1}^{\infty} X \{ |f| \geq t + \frac{1}{n} \}$
            \end{proof}
        }
        \item {
            $\int_X {|f|^p d \mu} = \int_{0}^{+\infty} {p t^{p-1} \mu X\{ |f| \geq t \} dt}$ при $p > 0$.

            \begin{proof}
                $\int_X {|f|^p d \mu} = \int_{0}^{+\infty} {\mu X \{ |f|^p \geq t \}dt} = \int_0^{+\infty}{\mu X \{ |f| \geq t^{\frac{1}{p}} \} dt} = \int_{0}^{+\infty}{g (t^{\frac{1}{p}}) dt} = \int_{0}^{+\infty}{p s^{p-1} g(s) ds}$
                
                Где $t = s^p$, $s = t^{\frac{1}{p}}$, $dt = ps^{p-1}ds$.
            \end{proof}
        }
    \end{enumerate}
\end{consequence}


% Module 2
\Subsection{Замена переменной}
\begin{definition}
    $\Omega$ и $\tilde{\Omega} \subset \mathbb{R}^m$ -- открытые.

    $\Phi: \Omega \rightarrow \tilde{\Omega}$.

    $\Phi$ -- диффеоморфизм, если 

    \begin{enumerate}
        \item $\Phi$ -- биекция.
        \item $\Phi$ -- непр. дифф.
        \item $\Phi^{-1}$ -- непр. дифф.
    \end{enumerate}
\end{definition}

\begin{remark}
    $Id = \Phi^{-1} \circ \Phi \implies x = (\Phi(x)^{-1})' \cdot (\Phi(x)) \cdot \Phi'(x) \implies 1 = det(\Phi^{-1})' (\Phi(x)) \cdot det(\Phi'(x))$.
\end{remark}

\begin{remark}
    Обозначение.

    $J_{\Phi} := det \Phi'$

    якобиан $=$ определитель матрицы Якоби.
\end{remark}

\begin{theorem} (о замене переменной).

    $\Phi: \Omega \rightarrow \tilde{\Omega}$ диффеоморфизм. $\Omega, \tilde{\Omega} \subset \mathbb{R}^m$ откр., $f: \tilde{\Omega} \rightarrow \tilde{\mathbb{R}}, \ f \geq 0$ измеримая. Тогда
    
    $\int_{\tilde{\Omega}}{f d \lambda_m} = \int_{\Omega}{f(\Phi(x)) |J_{\Phi}(x)| d \lambda_m}$.

    Такая же формула есть и для суммир. функций $f$.

    Частные случаи:
    \begin{enumerate}
        \item {
            Сдвиг: $\Phi(x) = x + a, \ a \in \mathbb{R}^m$.

            $\int_{\mathbb{R}^m}{f d \lambda_m} = \int_{\mathbb{R}^m} {f(x + a) d \lambda_m(x)}$
        }
        \item {
            $L: \mathbb{R}^m \rightarrow \mathbb{R}^m$ обратимое линейное отображение.

            $\int_{\mathbb{R}^m} {f d \lambda_m} = \int_{\mathbb{R}^m} {f (L x) |det L| d \lambda_m(x)}$
        }
        \item {
            Гомотетия: $Lx = c \cdot x, \ c \in \mathbb{R}, \ c > 0$.

            $\int_{\mathbb{R}^m}{f d \lambda_m} = c^m \cdot \int_{\mathbb{R}^m} {f(c\cdot x) d \lambda_m(x)}$.
        }
    \end{enumerate}
\end{theorem}

\begin{lemma} (о расщеплении).

    $\Phi: \Omega \rightarrow \tilde{\Omega}$ -- диффеоморфизм, $\Omega, \tilde{\Omega} \subset \mathbb{R}^m$ -- открытые, $a \in \Omega$, $1 \leq k \leq m - 1$.
    
    Тогда существует $U_a$ и $\Phi_2: U_a \rightarrow \mathbb{R}_m$, $\Phi_1: \Phi_2(U_a) \rightarrow \mathbb{R}^m$, т.ч. $\Phi = \Phi_1 \circ \Phi_2$.

    $\Phi_1$ -- осталяет на месте $k$ координат, а $\Phi_2$ -- оставляет на месте $m - k$ координат.
\end{lemma}
\begin{proof}
    $x, u \in \mathbb{R}^m, \ y, v \in \mathbb{R}^{m - k}, \ \Phi(x, y) = \left( \underbrace{\phi(x, y)}_{\in \mathbb{R}^{k}}, \ \underbrace{\psi(x, y)}_{\in \mathbb{R}^{m - k}} \right)$.

    $\Phi_1(x, y) = (x, \ \underbrace{f(x, y)}_{\in \mathbb{R}^{m - k}})$
    
    $\Phi_2(x, y) = (\underbrace{g(x, y)}_{\in \mathbb{R}^{k}}, \ y)$

    $\Phi_1(\Phi_2(x, y)) = (*)$
    
    $(*) = \Phi_1(g(x, y), y) = \left(g(x, y), f(g(x, y), y)\right)$

    $(*) = \left( \phi(x, y), \psi(x, y) \right) \implies g(x, y) := \phi(x, y)$
    
    
    $\implies f(u, v) = \psi(\Phi_2^{-1}(u, v))$
    
    $f(\phi_2(x, y)) = f(\phi(x, y), y) = \psi(x, y)$

    Нужна локальная обратимость $\Phi_2$,  а для этого нужна обратимость $\Phi_2'(a)$, то есть $det(\Phi_2'(a)) \not = 0$.

    $\Phi_2(x, y) = \left( \phi(x, y), y \right), \ \Phi_2'(x, y) = 
    \begin{pmatrix}
        \phi_x' & \phi_y' \\
        0 & E
    \end{pmatrix}
    , \ det (\Phi_2') = det(\Phi_x)$.

    $\Phi(x, y) = \left( \phi(x, y), \ \psi(x, y) \right)$

    $\Phi' =
    \begin{pmatrix}
        \phi_x' & \phi_y' \\
        \psi_x' & \psi_y'
    \end{pmatrix}
    $

    блок $k \times k$, ненулевой минор найдется.
\end{proof}

\begin{consequence}
    $\Phi: \Omega \rightarrow \tilde{\Omega}$ -- диффеоморфизм, $a \in \Omega, \ \Omega, \tilde{\Omega} \subset \mathbb{R}^m$ -- открытые.

    Тогда существует $U_a$, т.ч. $\Phi|_{U_a} = \Phi_1 \circ \Phi_2 \circ \dots \Phi_m$, где $\Phi_j$ -- диффеоморфизм, оставляющие на месте все координаты, кроме одной (но их перенумерующие).
\end{consequence}
\begin{proof}
    Индукция + предыдущая лемма.
\end{proof}

\begin{theorem}
    \textbf{Линделефа}.

    $A \subset \mathbb{R}^m, \ A$ -- покрыто открытыми мн-вами. 

    Тогда из него можно выделить не более чем счетное подпокрытие.
\end{theorem}
\begin{proof}
    $A \subset \bigcup_{\alpha \in I}{\left( \underbrace{G_{\alpha}}_{\text{открытое}} \right)}$.

    Берем $a \in A$, рисуем картинку, которую кто-нибудь \textit{обязательно} добавит.
    % todo: picture

    Пусть $U_a$ -- шарик с рациональным центром и рациональным радиусом. $a \in U_a$ и $U_a$ содержатся в каком-то элементе покрытия. Очевидно, что $a \in U_a \subset G_{\alpha_i}$, тогда выкинем все лишние $G_{\alpha}$, а остальных останется не более чем счетное кол-во (так как $U_a$ с рацинальным центром и радиусом, а таких счетное кол-во), при этом они покрывают $A$.
\end{proof}

\begin{theorem} (об изменении меры множества при диффеоморфизме).

    $\Phi: \Omega \rightarrow \tilde{\Omega}$ -- диффеоморфизм, $\Omega, \tilde{\Omega} \subset \mathbb{R}^m$ -- открытые, $A \subset \Omega$ -- измеримое. 

    Тогда $\lambda_m \Phi(A) = \int_{A}{|J_{\Phi}| d \lambda_m}$.
\end{theorem}
\begin{remark}
    Если теорема верна для конкретного $\Phi$ и произвольного $A$, то для того же $\Phi$ верна формула замена переменной.
    
    Формула замены переменной:
    
    $\int_{\tilde{\Omega}} {f d \lambda_m} = \int_{\Omega} {f \circ \Phi |J_{\Phi}| d \lambda_m}$.
\end{remark}
\begin{proof} Замечания.

    $f = \mathds{1}_{\Phi(A)}, \  A \subset \Omega$.

    $\int_{\tilde{\Omega}} {f d \lambda_m} = \int_{\tilde{\Omega}} {\mathds{1}_{\Phi(A)} d \lambda_m} = \Phi(A) = \int_{A}{|J_{\Phi}| d \lambda_m} = \int_{\Omega}{\mathds{1}_{A} |J_{\Phi}| d \lambda_m}$.

    % todo: picture (graph)

    $\mathds{1}_{\Phi(A)}(\Phi(x)) = \mathds{1}_A$.

    Нужно проверить для простых, а дальше для измеримых, в общем, все раскручивается (так говорил Храбров...).
\end{proof}
\begin{proof}
    Теоремы.

    \begin{itemize}
        \item [Шаг 1.] {
            Пусть $\Omega \subset \bigcup_{\alpha \in I} {G_{\alpha}}$. Если т. верна для каждого $G_{\alpha}$, то она верна и для $\Omega$.

            Выбираем нбчс подпокрытие $\Omega \subset \bigcup_{k = 1}^{\infty} {G_k}$.

            $\lambda_m \Phi \left(A \cap G_k\right) = \int_{A \cap G_k}{|J_{\Phi}| d \lambda_m}$ и просуммируем $A \cap \left( G_k \setminus \bigcup_{j = 1}^{k - 1} G_j \right)$.
        }
        \item [Шаг 2.] {
            Если т. верна для диффеоморфизмов $\Phi$ и $\Psi$, то она верна и для $\Psi \circ \Phi$.

            $\lambda_m \Psi(\Phi(A)) = \int_{\Phi(A)} {|J_{\Psi}| d \lambda_m} = \int_{\tilde{\Omega}} {\underbrace{\mathds{1}_{\Phi(A)} \cdot |J_{\Psi}|}_{=: \ f} d \lambda_m} =$
            
            $= \int_{\Omega} {\underbrace{\mathds{1}_{\Phi(A)} \circ \Phi}_{= \ \mathds{1}_A} \cdot  |J_{\Psi} \circ \Phi| \cdot |J_{\Phi}| d \lambda_m} =$
            
            $ = \int_A{|J_{\Psi} (\Phi(x))| |J_{\Phi}(x)| d \lambda_m(x)}$.

            $det(\Psi'(\Phi(x))) \cdot det(\Phi'(x)) = det \left( \Psi'(\Phi(x)) \cdot \Phi'(x) \right) = det(\Psi \circ \Phi)' = J_{\Psi \circ \Phi}$.
        }
        \item [Шаг 3.] {
            $m = 1$. $\Phi(x)$ -- строго монот. и непр. дифф.
            
            $\nu A := \lambda_1 (\phi(A))$ -- мера.
            
            $\mu A := \int_{A} {|\phi'| d \lambda_1}$ -- мера.

            Хотим проверить, что $\nu = \mu$, тогда проверим, что они совпадают на ячейках $(a, b]$ (а по единственности продолжения получим, что нужно).

            $\lambda(\phi(a, b]) = \int_{(a, b]}{|\phi'| d \lambda}$.

            Эти значения стремятся к тем, что выше, соответственно.
            $\lambda(\phi[a + \frac{1}{n}, b]) = \int_{[a + \frac{1}{n}, b]} {|\phi'| d \lambda}$

            Эти равны тем, что выше, соответственно.
            $\phi(b) - \phi(a + \frac{1}{n}) = \int_{a + \frac{1}{n}}^{b} {\phi' d \lambda}$, если $\phi$ -- возрастает, $\phi[a + \frac{1}{n}, b] = [\phi(a + \frac{1}{n}), \phi(b)]$
        }
        \item [Шаг 4.] {
            $\Phi$ оставляет на месте $m - 1$ коорд. $x = (\underbrace{y}_{\in \ \mathbb{R}^{m-1}}, \underbrace{t}_{\in \ \mathbb{R}})$.

            $\Phi(y, t) = \left( y, \phi(y, t) \right)$.

            $\lambda_m \Phi(A) = \int_{\mathbb{R}^{m - 1}} \left( \lambda_1 \Phi(A) \right)_y d \lambda_{m - 1}(y) = \int_{\mathbb{R}^{m - 1}} {\lambda_1 \left( \phi(y, A_y) \right) d \lambda_{m - 1}(y)} \underbrace{=}_{(*)} $.

            $t \in \left( \Phi(A) \right)_y \Leftrightarrow (y, t) \in \Phi(A) \Leftrightarrow \exists (y', t') \in A$, т.ч. $(y, t) = \Phi(y', t') = (y', \phi(y', t')) \Leftrightarrow \exists t': \ \underbrace{(y, t') \in A}_{t' \in A_y}$ и $\underbrace{(y, t) = (y, \phi(y, t'))}_{t = \phi(y, t')} \Leftrightarrow t \in \phi(y, A_y)$. 

            $\underbrace{=}_{(*)} \int_{\mathbb{R}^{m - 1}} {\left( \int_{A_y}{|\phi'(y, t)| d \lambda_1(t)} d \lambda_{m - 1}(y) \right)} = \int_A{| J_{\Phi} | \lambda_m}$.

            $
            \Phi' =
            \begin{pmatrix}
                E & 0 \\
                \phi_y' & \phi_t'
            \end{pmatrix}
            $

            Дальше были какие-то умные слова. Я не успел записать...
        }
    \end{itemize}
\end{proof}

\begin{example}
    Полярная замена. $\mathbb{R}^2$.

    $(r, \phi) \rightarrow (r \cos{(\phi)}, r \sin{(\phi)})$

    $r \in (0, +\infty)$

    $\phi \in (0, 2 \pi)$


    $\int_{\mathbb{R}^2}{f(x, y) d \lambda_2} = \int_{[0, 2 \pi] \times [0, +\infty)} {\left( f(r \cos{(\phi)}, r \sin{(\phi)}) \cdot r \right)} d r d \phi$.

    $\Phi' = 
    \begin{pmatrix}
        \frac{dx}{dr} & \frac{dx}{d\phi} \\
        \frac{dy}{dr} & \frac{dy}{d\phi}
    \end{pmatrix}
    =
    \begin{pmatrix}
        \cos{(\phi)} & -r \sin{(\phi)} \\
        \sin{(\phi)} & r \cos{(\phi)}
    \end{pmatrix}
    $

    $det = r$

    $\int_{\mathbb{R}} {e^{-x^2} d x}, \ f(x, y) = e^{-x^2 - y^2}$

    $\int_{\mathbb{R}^2}{e^{-x^2 - y^2} dx dy} = \int_{\mathbb{R}} {e^{-x^2} dx} \cdot \int_{\mathbb{R}} {e^{-y^2} dy} = \left( \int_{\mathbb{R}} {e^{-x^2} dx} \right)^2$.

    Полярная замена:

    $\int_{0}^{+\infty} { \int_{0}^{2 \pi} { e^{-r^2} r d \phi d r } } = 2\pi \int_{0}^{+\infty} { e^{-r^2} r d r } = \pi \int_{0}^{+\infty} {e^{-t} d t} = \pi \cdot (-e^{-t})|_{0}^{+\infty} = \pi$.

    $t = r^2, \ df = 2r d r$
\end{example}

