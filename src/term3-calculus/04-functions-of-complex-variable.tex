\Subsection{Голоморфные функции}

Если доказательство не указано, то оно повторяет то, что было в $\mathbb{R}$ (смотреть 1 семестр).

\begin{definition}
    $\Omega$ -- обсласть в $\mathbb{C}, \ f: \Omega \rightarrow \mathbb{C}, \ z_0 \in \Omega$.

    $f$ -- голоморфна в точке $z_0$, если существует $\lim_{z \rightarrow z_0} { \frac{f(z) - f(z_0)}{z - z_0} } =: f'(z_0)$.
\end{definition}
\begin{definition}
    $f$ комплексно дифф. в точке $z_0$, если $\exists k \in \mathbb{C}$:

    $f(z) = f(z_0) + k (z - z_0) + o(z - z_0)$ при $z \rightarrow z_0$.
\end{definition}

\begin{statement}
    $f$ -- голоморфна в точке $z_0 \Leftrightarrow f$ комплексно дифф. в точке $z_0$ и $k = f'(z_0)$.
\end{statement}

\begin{consequence}
    $f$ и $g$ голоморфны в точке $z_0$. Тогда 

    \begin{enumerate}
        \item {
            $f \pm g$ голом. в точке $z_0$
        }
        \item {
            $f \cdot g$ голом. в точке $z_0$
        }
        \item {
            Если $g(z_0 \not = 0)$, то $\frac{f}{g}$ голом. в точке $z_0$.
        }
        \item {
            Если $h$ голом. в точке $f(z_0)$, то $h \circ f$ голом. в точке $z_0$.
        }
    \end{enumerate}
\end{consequence}

\begin{remark}
    $f: \Omega \rightarrow \mathbb{C}$

    $z = x + iy, \ f(z) = f(x + iy) = g(x + i y) + i h(x + iy): \ g, h : \Omega \rightarrow \mathbb{R}$.

    $\frac{\partial f}{\partial x} (z_0) = \lim_{h \rightarrow 0, \ h \in \mathbb{R}} {\frac{f(z_0 + h) - f(z_0)}{h}} = f'(z_0)$.

    
    $\frac{\partial f}{\partial y} (z_0) = \lim_{h \rightarrow 0, \ h \in \mathbb{R}} {\frac{f(z_0 + i h) - f(z_0)}{h}} = \frac{f'(z_0)}{i} = -i \cdot f'(z_0)$.
\end{remark}

\begin{remark}
    $\binom{g(x + iy)}{h(x + iy)} = \binom{g(x_0 + i y_0)}{h(x_0 + iy_0)} + \binom{a \ b}{c \ d} \binom{a - x_0}{y - y_0} + o(|| (x - x_0, y - y_0) ||)$.

    $k = \alpha + i \beta$

    $k \cdot (z - z_0) = (\alpha + i \beta) ( (x - x_0) + i (y - y_0)) = \alpha(x - x_0) - \beta(y - y_0) + i (\beta (x - x_0) + \alpha (y - y_0))$

    Вещественная линейность + $\binom{\alpha \ -\beta}{\beta \ \alpha} \Leftrightarrow$ комплескная линейность.
\end{remark}


% TODO: где-то есть места, где я пишу `\pi` и при этом забыто `\cdot i`, то есть надо `\pi \cdot i`. Это критическая ошибка, ее надо исправить !!!

\begin{remark}
    Комплескная дифференцируемость $\Leftrightarrow$ вещественная дифференцируемость + матрица Якоби $\binom{\alpha \ \beta}{-\beta \ \alpha}$

    
    Комплескная дифференцируемость $\Leftrightarrow$ вещественная дифференцируемость + условия Коши-Римана $
    \begin{cases}
        \frac{\partial Re(f)}{\partial x} = \frac{\partial Im (f)}{\partial y} \\ 
        \frac{\partial Re(f)}{\partial y} = \frac{\partial Im (f)}{\partial x} \\ 
    \end{cases}
    $
\end{remark}
\begin{remark}
    $f(z) = f(z_0) + \underbrace{k}_{\in \mathbb{C}} (z - z_0) + o(z - z_0)$

    $k (z - z_0) = k w = |k| \cdot e^{i \phi} \cdot w, \ \phi = arg(k)$
\end{remark}


\begin{remark}
    Обозначения.

    $\frac{\partial}{\partial z} = \frac{1}{2} \cdot \left( \frac{\partial}{\partial x} - i \frac{\partial}{\partial y} \right)$
    
    $\frac{\partial}{\partial \overline{z}} = \frac{1}{2} \cdot \left( \frac{\partial}{\partial x} - i \frac{\partial}{\partial y} \right)$

    $dz = dx + i dy$

    $d \overline{z} = dx - i dy$

    $df = \frac{\partial f}{\partial x} \cdot dx + \frac{\partial f}{\partial y} dy = \frac{\partial f}{\partial z} dz + \frac{\partial f}{\partial \overline{z}} d \overline{z}$
\end{remark}

\begin{theorem}
    \textbf{Условия Коши-Римана}.

    $f: \Omega \rightarrow \mathbb{C}, \ a \in \Omega$

    $f$ -- дифф. в точке $a$ как функция из $\mathbb{R}^2$ в $\mathbb{R}^2$. Следующие условия равносильны:

    \begin{enumerate}
        \item {
            $f$ -- голоморфна в точке $a$.
        }
        \item {
            $d_a f$ -- комплексно линеен
        }
        \item {
            условия Коши-Римана
        }
        \item {
            $\frac{\partial f}{\partial \overline{z}} (a) = 0$
        }
    \end{enumerate}
\end{theorem}

\begin{proof}
    Мы выяснили все, кроме $(3) \Leftrightarrow (4)$:

    $\frac{\partial f}{\partial \overline{z}} = 0 \Leftrightarrow \frac{\partial f}{\partial x} + i \frac{\partial f}{\partial y} = 0 \Leftrightarrow \frac{\partial (Re(f) + i Im(f))}{\partial x} + i \cdot \frac{\partial (Re (f) + i Im (f))}{\partial y} = 0 \Leftrightarrow \begin{cases}
        \frac{\partial Re(f)}{\partial x} - \frac{\partial Im (f)}{\partial y} = 0 \\ 
        \frac{\partial Im(f)}{\partial x} + \frac{\partial Re(f)}{\partial y} = 0
    \end{cases}$ -- а это и есть условия Коши-Римана.
\end{proof}

\begin{remark}
    Обозначения.

    $f \in H(\Omega) \Leftrightarrow f : \Omega \rightarrow \mathbb{C}$ и голоморфна во всех точках из $\Omega$.
\end{remark}
\begin{consequence}
    $\Omega$ -- область, $f \in H(\Omega)$ и $Im(f) = const \implies f = const$
\end{consequence}
\begin{proof}
    $\frac{\partial Im (f)}{\partial y} = 0 \implies \frac{\partial Re(f)}{\partial x} = 0$

    $\frac{\partial Im (f)}{\partial x} = 0 \implies \frac{\partial Re(f)}{\partial y} = 0$

    $\implies Re(f) = const$
\end{proof}
\begin{theorem}
    \textbf{Коши} (ah, shit, here we go again...)

    $f \in H(\Omega) \implies f(z) dz$ локально точная.
\end{theorem}
\begin{proof}
    Будет два разных док-ва.

    \begin{enumerate}
        \item {
            Для случая непрерывно-дифф. $\frac{\partial Re(f)}{\partial x}, \dots$ (имеются в виду все частные производные).

            Тогда замкнутость $\implies$ локальная точность.

            $f(z) dz = f(z) (dx + i dy) = (Re(f) + i \cdot Im(f)) \cdot (dx + i dy) = Re(f) dx - Im(f) dy + i (Im (f) dx + Re(f) dy)$.

            $P dx + Q dy$ -- замкн. $\Leftrightarrow \frac{\partial P}{\partial y} = \frac{\partial Q}{\partial x}$

            $Re(f) dx - Im(f) dy$ -- замкн. $\Leftrightarrow \frac{\partial Re(f)}{\partial y} = - \frac{\partial Im(f)}{\partial x}$

            $Im(f) dx + Re(f) dy$ -- замкн. $\Leftrightarrow \frac{\partial Im(f)}{\partial y} = \frac{\partial Re(f)}{\partial x}$
        }
        \item {
            Общий случай.

            % todo: picture
            Надо доказать, что интеграл по любому прямоугольнику (со сторонами параллельными осям) из круга (круг берем вокруг точки из $\Omega$. Добавьте картинку, плиз) равен 0.

            От противного: пусть нашелся прямоугольник $P$, т.ч. $\alpha (P) := \int_P {f(z) dz } \not = 0$.

            % todo: picture
            Режем прямоугольник на 4 части, индексируем как $P^{1}, P^{2}, P^{3}, P^{4}$, строим объоды каждого (против часовой стрелки). Тогда $\alpha(P) = \alpha(P^{1}) + \alpha(P^{2}) + \alpha(P^{3}) + \alpha(P^{4})$, $|\alpha(P)| \leq |\alpha(P^{1})| + |\alpha(P^{2})| + |\alpha(P^{3})| + \alpha(P^{4})$.

            Хотя бы одно из слагаемых $\geq \frac{1}{4} |\alpha(P)|$, назовем такое $P_1$ (индекс уже снизу!). Разрежем его на 4 равные части. Пусть $P_2$ такой, что $|\alpha(P_2)| \geq \frac{1}{4} |\alpha (P_1)|$ и т.д.

            $|\alpha(P_n)| \geq \frac{1}{4^n} |\alpha (P)|$.

            % todo: picture
            Берем $a$ из $P_n$:

            $f(z) = f(a) + f'(a) (z - a) + o(z - a)$

            $\alpha (P_n) = \int_{P_n} { f(z) dz } = \underbrace{\int_{P_n} { f(a) dz }}_{= 0} + \underbrace{\int_{P_n} { f'(a) (z - a) dz }}_{= 0} + \int_{P_n} { o(z - a) dz }$

            $o(z - a) = (z - a) \cdot \beta (z - a)$, где $\beta(z - a) \underbrace{\rightarrow}_{z \rightarrow a} 0$

            $\left| \int_{P_n} {(z - a) \beta (z - a) dz} \right| \leq max_{z \in P_n} { |z - a| \cdot |\beta (z - a)| } \cdot \underbrace{l(P_n)}_{\text{периметр}} \leq max_{z \in P_n} { |\beta (z - a)| } \cdot \frac{l(P)}{2^n} \cdot \frac{c}{2^n} \implies$

            $\implies \frac{|\alpha (P)|}{4^n} \leq |\alpha(P_n)| \leq \frac{l(P) \cdot c}{4^n} \cdot max_{z \in P_n} |\beta (z - a)| \implies max_{z \in P_n} |\beta (z - a)| \geq \frac{|\alpha (P)|}{l(P) \cdot c} > 0$ -- противоречие.
        }
    \end{enumerate}
\end{proof}
\begin{consequence}
    \begin{enumerate}
        \item {
            Если $f \in H(\Omega)$, то у каждой точки $a \in \Omega$ есть окрестность, в которой существует ф-я $F$, т.ч. $F' = f$ в этой окрестности.

            \begin{proof}
                Пусть $F$ первообразная формы $f(z) dz$. Поймем, что $F' = f$.

                $\frac{\partial F}{\partial x} = f(z), \ \frac{\partial F}{\partial y} = i \cdot f(z) \implies \frac{\partial F}{\partial x} + i \frac{\partial F}{\partial y} = 0 \implies \frac{\partial F}{\partial \overline{z}} = 0$
            \end{proof}
        }
        \item {
            $f \in H(\Omega)$, $\gamma$ стягиваемый в $\Omega$ путь $\implies \int_{\gamma} { f(z) dz } = 0$
        }
    \end{enumerate}
\end{consequence}
\begin{theorem}
    $f \in C(\Omega), \ \Delta$ -- прямая параллельная оси координат.

    $f \in H(\Omega \setminus \Delta)$

    Тогда $f(z) dz$ локально точная.
\end{theorem}
\begin{proof}
    Надо проверять, что интеграл по довольно маленькому прямоугольнику (со стороронами паралл. осям) это 0.

    % todo: picture!!!
    Очевидно, что если прямоугольник не пересекает $\Delta$, то там все очевидно. Хотим рассматривать только те, что задевают. Те, что пересекают $\Delta$, можно разбить на две части (верхнюю и нижнюю). По каждой из частей будет 0, тогда и в сумме тоже будет 0. То есть нас вообще интересуют только те прямоугольники, у которых $\Delta$ это одна из сторон. Рассмотрим их:
    
    % todo: picture

    Тут мастхэв картинка, на которой мы откусывает подпрямугольник размера $\epsilon$.

    $\int_{P_{\epsilon}} { f(z) d z } = 0 \rightarrow_{\epsilon \rightarrow 0} \int_{P} { f(z) dz }$

    $\left|\int_{P} {f(z) dz} - \int_{P_{\epsilon}} { f(z) dz } \right| \leq |\int_{1} + \int_{3}| + |\int_{2}| + |\int_{4}|$

    $\left| \int_{2} {f(z) dz} \right| \leq M \cdot (\text{длина } 2) = M \epsilon$

    $\left| \int_{1} + \int_{3} \right| = \left| \int_{a}^{b} { \left(f (x + i y_0) - f(x + i(y_0 + \epsilon)) \right) dx } \right| \leq \int_{a}^{b} { |\dots| dx } = (*)$

    $f$ непрер. на компакте $\implies$ равномерно непрер.

    $\forall \gamma > 0: \ \exists \epsilon > 0$ если $\rho (\text{аргумент}) < \epsilon \implies |f(\dots) - f(\dots)| < \gamma$, тогда 

    $(*) \leq (b - a) \cdot \gamma$
\end{proof}

\begin{consequence}
    $f: \Omega \rightarrow \mathbb{C}$

    $f \in C(\Omega)$ и $f$ голоморфна в $\Omega$ за исключением мн-ва изолированных точек, тогда форма $f(z) dz$ все равно лок. точная.
\end{consequence}
\begin{proof}
    Рассмотрим окр-ть, в которой ровно одна плохая точка.

    % todo: picture
    Давайте проведем прямую через это точку, тогда работает теорема.
\end{proof}

\begin{definition}
    Индекс кривой отн-но точки $Ind(\gamma, z_0)$.

    $\gamma$ -- замкнутая кривая, не проходящая через точку $z_0$.

    $Ind(\gamma, 0) = \frac{\phi(b) - \phi(a)}{2\pi} \in \mathbb{Z}$ -- кол-во оборотов $\gamma$ вокруг 0.

    $\gamma: [a, b] \rightarrow \mathbb{C}$

    $\gamma(t) = r(t) e^{i \phi(t)}$, $\phi$ -- непрерывна (полярная замена).
\end{definition}

\begin{theorem}
    Пусть $\gamma$ -- замкнутая кривая, не проходящая через 0. Тогда
    
    $\int_{\gamma} { \frac{\partial z}{z} } = 2\pi Ind(\gamma, 0)$.
\end{theorem}
\begin{proof}
    Берем параметризацию $r, \phi: [a, b] \rightarrow \mathbb{R}$

    $z(t) = r(t) e^{i \phi(t)}, \ dz = \left( r' e^{i\phi} + ri \phi' e^{i\phi} \right) dt$

    $\frac{dz}{z} = \frac{r'}{r} + i \phi'$

    $\int_{\gamma} { \frac{dz}{z} } = \int_{a}^{b} { \left( \frac{r'(t)}{r(t)} + i \phi'(t) \right) dt } = \left( ln (r(t)) + i \phi(t) \right)|_{t = a}^{t = b} = i (\phi(b) - \phi(a)) = 2 \pi Ind(\gamma, 0)$
\end{proof}

\begin{consequence}
    Пусть $\gamma$ -- замкнутая кривая, не проходящая через точку $a$. Тогда

    $\int_{\gamma} { \frac{dz}{z - a} = 2 \pi Ind (\gamma, a) }$.
\end{consequence}

\begin{theorem}
    (интегральная формула Коши).

    $f \in H(\Omega)$

    $\gamma$ -- стягиваемая в $\Omega$ кривая, не проходящая через $a \in \Omega$.

    Тогда $\int_{\gamma} { \frac{f(z) dz}{z - a} } = 2 \pi f(a) Ind(\gamma, a)$
\end{theorem}
\begin{proof}
    $g(z) = \begin{cases}
        \frac{f(z) - f(a)}{z - a}, \ \text{при } z \not = a, \\
        f'(a), \ \text{иначе}
    \end{cases}$

    $g \in C(\Omega)$

    $g \in H(\Omega \setminus \{ a \})$

    $\implies g(z) dz$ -- локально точкая форма $\implies \int_{\gamma} {g(z) dz} = 0$, так как $\gamma$ -- стягиваемая

    $\implies 0 = \int_{\gamma} { \frac{f(z) dz}{z - a} } - \int_{\gamma} { \frac{f(a) dz}{z - a} } \implies \int_{\gamma} { \frac{f(z) dz}{z - a} } = f(a) \cdot \int_{\gamma} { \frac{dz}{z - a} } = f(a) \cdot 2 \pi \cdot Ind(\gamma, a)$
\end{proof}
\begin{example}
    % todo: picture !!!
    Берем круг. $f$ -- голоморфна в окр-ти этого круга.

    $\int_{\text{окр.}} { \frac{f(z)}{z - a} dz } = \begin{cases}
        0, \ \text{ если $a$ вне круга} \\
        f(a) \cdot 2 \pi, \ \text{ если $a$ внутри круга }
    \end{cases}$
\end{example}

\begin{remark}
    Обозначение.

    $\mathbb{D} = \{ |z| < 1 \}$ -- единичный круг.

    $\mathbb{T} = \{ |z| < 1 \}$ -- единичная окружность, обход против часовой стрелки.

    $r\mathbb{T} + a = \{ |z - a| = r \}$
\end{remark}

\begin{theorem}
    $f \in H(r \mathbb{D}) \implies f$ аналитична ($=$ функция раскладывается в ряд) в этом круге.
\end{theorem}

\begin{proof}
    % todo: picture
    В наше круге радиуса $r$ берем еще два круга с тем же центром, но меньшими радиусами ($r > r_1 > r_2 > 0$). Берем $z: \ |z| < r_2$ -- точка внутри наименьшего круга. Хотим интегрировать по средней окружности.

    $f(z) = \frac{1}{2 \pi} \int_{r_1 \mathbb{T}} { \frac{f(\zeta) d \zeta }{\zeta - z} }$

    $\frac{1}{\zeta - z} = \frac{1}{1 - \frac{z}{\zeta}} \cdot \frac{1}{\zeta} = \sum_{n=0}^{\infty} \frac{z^n}{\zeta^{n+1}} = (*)$ равномерно сх-ся, так как $\left| \frac{z}{\zeta} \right| \leq \frac{r_2}{r_1}$

    $(*) = \frac{1}{2\pi} \int_{r_1 \mathbb{T}} { \sum_{n=0}^{\infty} \frac{f(\zeta)}{\zeta^{n+1}} z^n d\zeta } = \frac{1}{2\pi} \sum_{n=0}^{\infty} z^n \underbrace{\int_{r_1 \mathbb{T}} {\frac{f(\zeta)}{\zeta^{n+1}} d\zeta}}_{=: a_n \cdot 2\pi} = \sum_{n=0}^{\infty} {a_n z^n}$
\end{proof}



\begin{consequence}
    \begin{enumerate}
        \item {
            Если $f \in H(r \mathbb{D})$ и $0 < r_1 < r$, то
    
            $\frac{n!}{2 \pi} \cdot \int_{r_1 \mathbb{T}} { \frac{f(z)}{z^{n+1}} dz } = f^{(n)} (0)$
        }
        \item {
            $f \in H(r \mathbb{D} + a), \ 0 < r_1 < r \implies \frac{n!}{2 \pi} \int_{r_1 \mathbb{T} + a} { \frac{f(z)}{(z - a)^{n + 1}} dz } = f^{(n)} (a)$

            $z = w + a$

            $g(w) = f(w + a)$

            $g^{(n)} (0) = \frac{n!}{2 \pi} \cdot \int_{r_1 \mathbb{T}} { \frac{g(w)}{w^{n + 1}} dw }$
        }
        \item {
            $f: \Omega \rightarrow \mathbb{C}$

            Тогда $f$ -- голоморфна в $\Omega \Leftrightarrow f$ -- аналитична в $\Omega$.

            % todo: picture
        }
        \item {
            $f \in H(\Omega) \implies f$ -- бесконечно диффиренцируема.
        }
        \item {
            $f \in H(\Omega) \implies f' \in H(\Omega)$
        }
        \item {
            \begin{definition}
                $g: \mathbb{R}^n \rightarrow \mathbb{R}$ -- гармоническая, если $\frac{\partial^2 g}{\partial x_1^2} + \frac{\partial^2 g}{\partial x_2^2} + \dots + \frac{\partial g}{\partial x_n^2} = 0$.
            \end{definition}
            
            Продолжаем свойство:

            $f \in H(\Omega) \implies Re (f)$ и $Im(f)$ -- гармонические функции.
            
            \begin{proof}
                $\frac{\partial^2 Re(f)}{\partial x^2} = \frac{\partial}{\partial x} \left( \frac{\partial Re (f)}{\partial x} \right) = \frac{\partial}{\partial x} \left( \frac{\partial Im(f)}{\partial y} \right) = \frac{\partial}{\partial y} \left( \frac{\partial Im(f)}{\partial x} \right) = \frac{\partial}{\partial y} \left( - \frac{\partial Re(f)}{\partial y} \right) = - \frac{\partial^2 Re(f)}{\partial y^2}$

                про $Im(f)$ аналогично доказывается.
            \end{proof}
        }
    \end{enumerate}
\end{consequence}

\begin{remark}
    Если $g: \Omega \rightarrow \mathbb{R}$ гармоническая ф-я, то существует единств. (с точностью до прибавления $const \in \mathbb{R}$) гармоническая ф-я $h: \Omega \rightarrow \mathbb{R}$, т.ч. $g + i h \in H(\Omega)$
\end{remark}

\begin{theorem}
    \textbf{Мореры}.

    $f \in C(\Omega)$. Если $f(z) dz$ локально точная, то $f \in H(\Omega)$.
\end{theorem}
\begin{proof}
    Возьмем $a \in \Omega$. Существует окр-ть $a$, что для $f$ в ней есть первообразная $F$ (т.е. $F' = f$ в $U$). 
    
    Тогда $F \in H(U) \implies F' = f \in H(U)$ -- это локальное свойство, поэтому на всей $\Omega$ тоже будет гомоморфность.
\end{proof}

\begin{consequence}
    $f \in C(\Omega), \ \Delta$ -- прямая, параллельная оси координат.

    $f \in H(\Omega \setminus \Delta)$. Тогда $f \in H(\Omega)$.
\end{consequence}
\begin{proof}
    $f \in C(\Omega)$ и $f \in H(\Omega \setminus \Delta) \implies f(z) dz$ локально точная в $\Omega \underbrace{\implies}_{\text{т. Мореры}} f \in H(\Omega)$.
\end{proof}

\begin{theorem}
    (интегральная формула Коши).

    $f \in H(\Omega)$

    $K \subset \Omega$ -- компакт, граница которого -- конечное число кусочно-гладких замкнутых кривых. Тогда 

    \begin{enumerate}
        \item {
            $\int_{\partial K} { f(z) dz } = 0$
        }
        \item {
            Если $a \in Int (K)$, то $\int_{\partial K} { \frac{f(z)}{z - a} dz } = 2 \pi f(a)$.
        }
    \end{enumerate}
\end{theorem}
\begin{proof}
    \begin{enumerate}
        \item {
            Пишем \textbf{формулу Грина}.

            $\int_{\partial K} { f(z) dz } = \int_{\partial K} { f(z) dx + i \cdot f(z) dy } \underbrace{=}_{\text{Грин}} \int_{K} { \left( i \cdot \frac{\partial f}{\partial x}  - \frac{\partial f}{\partial y} \right) dx dy } = $
            
            $ = i \cdot \int_{K} {\left( \frac{\partial f}{\partial x} + i \frac{\partial f}{\partial y} \right) dx dy} = 2 i \int_{K} { \frac{\partial f}{\partial \overline{z}} d \lambda_2 } = 0$.
        }
        \item {
            % todo: picture !!!
            Берем круг, содержащий $a$, не вылезающий за границу формы $B_r(a)$.
            
            $\tilde{K} = K \setminus B_r(a)$ -- компакт.

            $\frac{f(z)}{z - a} \in H(\Omega \setminus \{ a \}), \ \tilde{K} \subset \Omega \setminus \{ a \}$.

            $0 = \int_{\partial \tilde{K}} { \frac{z}{z - a} dz } = \int_{\partial K} { \frac{f(z)}{z - a} dz } - \underbrace{\int_{r \mathbb{T} + a} { \frac{f(z)}{z - a} dz }}_{= 2\pi i f(a)}$.
        }
    \end{enumerate}
\end{proof}
\begin{exerc}
    $f \in H(r \mathbb{D})$ и $f \in C(Cl (r \mathbb{D}))$

    $a \in \mathbb{D}$.
    
    Доказать, что $\int_{r \mathbb{T}} { \frac{f(z)}{z - a} dz } = 2 \pi i f(a)$
\end{exerc}

\begin{theorem}
    $f \in C(\Omega)$. Следующие условия равносильны (равносильность всех утверждений, так или иначе, уже доказывалась ранее):

    \begin{enumerate}
        \item {
            $f \in H(\Omega)$
        }
        \item {
            $f(z) dz$ -- локально точная в $\Omega$
        }
        \item {
            В окр-ти каждой точки у $f$ есть первообразная
        }
        \item {
            $f$ аналитична в $\Omega$
        }
        \item {
            $\int{f(z) dz} = 0$ по любому достаточно малому прямоугольнику со сторонами параллельными осям
        }
        \item {
            $f(z) dz$ -- замкнутая и частн. производные по $x$ и $y$ непрерывны.
        }
    \end{enumerate}
\end{theorem}

\begin{theorem}
    \textbf{Неравенство Коши}.

    $f \in H(R \mathbb{D}), \ 0 < r < R$.

    $f(z) = \sum_{n=0}^{\infty} {a_n z^n}$. Тогда $|a_n| \leq \frac{M(r)}{r^n}$, где $M(r) := \max_{|z| = r} |f(z)|$.
\end{theorem}
\begin{theorem}
    $a_n = \frac{1}{2 \pi i} \int_{|z| = r} { \frac{f(z)}{z^{n + 1}} dz }$

    $|a_n| = \frac{1}{2 \pi} \left| \int_{|z| = r} { \frac{f(z)}{z^{n+1}} dz } \right| \leq \frac{1}{2 \pi} \cdot \max_{|t| = r} \left|\frac{f(z)}{z^{n+1}}\right| \cdot 2 \pi r = \frac{M(r)}{r^{n + 1}} \cdot r = \frac{M(r)}{r^n}$
\end{theorem}

\begin{theorem}
    \textbf{Луивилля}.

    Если $f \in H(\mathbb{C})$ и $f$ -- ограничена, то $f = const$.
\end{theorem}
\begin{proof}
    $f$ -- ограничена $\implies |f| \leq M$.
    
    $f \in H(\mathbb{C}) \implies f(z) = \sum_{n=0}^{\infty} { a_n z^{n} }$ и ряд сходится $\forall z \in \mathbb{C} \underbrace{\implies}_{\text{нер-во Коши}} |a_n| \leq \frac{M_r}{r^n} \leq \frac{M}{r^n} \underbrace{\rightarrow}_{r \rightarrow +\infty} 0 \implies a_n = 0: \ \forall n \geq 1$
\end{proof}

\begin{remark}
    $\sin$ и $\cos$ неограничены в $\mathbb{C}$.
\end{remark}

\begin{definition}
    Целая функция -- функция, голоморфная в $\mathbb{C}$.
\end{definition}

\begin{theorem}
    \textbf{Основная теорема алгебры}.

    $P$ -- многочлен степени $\geq 1$. Тогда у $P$ есть хотя бы один корень.
\end{theorem}
\begin{consequence}
    Если $deg P = n$, то $P(z) = c(z - z_1)(z - z_2) \dots (z - z_n)$ для некоторых $z_1, z_2, \dots z_n \in \mathbb{C}$.
\end{consequence}
\begin{proof}
    Если $z_1$ -- корень $P$, то $P(z) = (z - z_1) \cdot Q(z)$, где $deg Q = n - 1$.
\end{proof}
\begin{proof}
    Основной теоремы алгебры.

    От противного:

    пусть $P(z) \not = 0 \ \forall z \in \mathbb{C}$. Тогда $f(z) = \frac{1}{P(z)} \in H(\mathbb{C})$.

    Докажем, что $f$ -- ограниченная функция.

    $P(z) = z^n + a_{n-1} z^{n-1} + \dots + a_1 z + a_0$

    $R := 1 + |a_{n-1}| + |a_{n-2}| + \dots + |a_1| + |a_0|$. Пусть $|z| \geq R, \ |P(z)| \geq |z|^n - |a_{n-1}| |z|^{n-1} - \dots - |a_1| |z| - |a_0| \geq |z|^n - |z|^{n-1} (|a_{n-1}| + |a_{n-2}| + \dots + |a_0|) = \underbrace{|z|^{n-1}}_{\geq 1} \underbrace{(|z| - |a_0| - |a_1| - \dots - |a_{n-1}|)}_{\geq 1} \implies |P(z)| \geq 1$ при $|z| \geq R \implies |f(z)| \leq 1$ при $|z| \geq R$.

    Докажем, что при $|z| \leq R, \ |f(z)|$ -- ограничена.

    $f \in H(\mathbb{C}) \implies f$ непрер. в $\mathbb{C} \implies f$ непрер. в $\{ |z| \leq R \}$ -- компакт $\implies |f|$ огр. в $\{ |z| \leq R \}$.
\end{proof}


\Subsection{Теоремы единственности}
\begin{theorem}
    $f \in H(\Omega)$, $\Omega$ -- область, $z_0 \in \Omega$. След. условия равносильны:

    \begin{enumerate}
        \item {
            $f^{(n)} (z_0) = 0 \ \forall n = 0, 1, 2, \dots$
        }
        \item {
            $f = 0$ в некоторой окр-ти точки $z_0$.
        }
        \item {
            $f \equiv 0$ в $\Omega$
        }
    \end{enumerate}
\end{theorem}


\begin{lemma}
    $\Omega$ -- область в метрическом пространстве, $E \subset \Omega$, т.ч. $E \not = \emptyset, \ E$ -- открыто в $\Omega$, $E$ -- замкнуто в $\Omega$. Тогда $E = \Omega$.
\end{lemma}
\begin{proof}
    Пусть $\Omega \setminus E \not = \emptyset$, берем $a \in E$ и $b \in \Omega \setminus E$. Возьмем путь $\gamma$, соединяющий эти точки.

    $\gamma: [\alpha, \beta] \rightarrow \Omega$, т.ч. $\gamma(\alpha) = a, \ \gamma(\beta) = b$. $\gamma$ -- непрер. $\implies \gamma^{-1} (E)$ -- открыто, $\gamma^{-1}(\Omega \setminus E)$ -- открыто $\implies \gamma^{-1}(E)$ -- открыт. и замкнут. подмн-во $[\alpha, \beta], \ \alpha \in \gamma^{-1}(E), \ \beta \not \in \gamma^{-1}(E)$.

    $s := \sup{\gamma^{-1} (E)}$ из замкн. $s \in \gamma^{-1} (E) \implies s < \beta$.

    % todo: picture

    Возьмем окр-ть $s$, т.ч. $(s - \delta, s + \delta) \subset \gamma^{-1}(E) \cap (\alpha, \beta) \implies $ в $\gamma^{-1}(E)$ есть точки $> s \implies s $ не $\sup$. Противоречие. 
\end{proof}

\begin{proof}
    Теоремы.

    $(3) \implies (2) \implies (1)$ -- очевидно.

    $(1) \implies (2)$ -- почти очевидно:

    % todo: picture
    Берем $z_0 \in \Omega$ и $B_r(z_0) \subset \Omega$, тогда в круге $|z - z_0| < r: $ $f$ раскл. в свой ряд Тейлора $\implies$ в нем $f \equiv 0$.

    $(2) \implies (3)$:

    $E := \{ z \in \Omega: \ \text{в некоторой окр-ти точки } z, \ f = 0 \}$

    $z_0 \in E$ по условию $\implies E \not = \emptyset$.

    $E$ -- открыто. Если $w \in E$, то в круге $|z - w| < r, \ f = 0$.
    % todo: picture

    $\forall z$ из этого круга есть круг меньшего радиуса, содерж. $\{ |z - w| < r \}$, в нем $f = 0$.

    $E$ -- замкнуто. Пусть $z_*$ -- предельная точка $E$, то есть $z_n \in E$ и $\lim{z_n} = z_*$. $f^{(m)} (z_n) = 0 \ \forall m, \ \forall n$ (так как есть $(2) \implies (1)$). По непрерывности $f^{(m)} \ f^{(m)} (z_*) = \lim{f^{(m)} (z_n)} = 0 \underbrace{\implies}_{(1) \implies (2)} z_* \in E$.

    Тогда по лемме $E = \Omega$.
\end{proof}


\begin{consequence}
    $f, g \in H(\mathbb{C})$, т.ч. $f(z) = g(z)$ в окр-ти точки $z_0 \in \Omega \implies f \equiv g$.
\end{consequence}



\begin{theorem}
    \textbf{О среднем}.

    $f \in H(\Omega)$ и $a \in \Omega$, причем $\{ |z - a| \leq r \} \subset \Omega$, тогда $f(a) = \frac{1}{2\pi} \cdot \int_{0}^{2\pi} { f(a + r e^{i \phi}) d \phi }$
\end{theorem}
\begin{proof}
    $f(a) = \frac{1}{2 \pi i}\int_{|z - a| = r}{ \frac{f(z)}{z - a} dz } = \frac{1}{2 \pi i} \int_{0}^{2\pi} { \frac{f(a + r e^{i \phi})}{r e^{i \phi}} r e^{i \phi} i d \phi }$, где $z = a+re^{i\phi}, \ dz = r e^{i \phi} i d \phi$.
\end{proof}

\begin{consequence}
    $f \in H(\Omega), \ a \in \Omega, \ \{ |z - a| \leq r \} \subset \Omega$. Тогда $f(a) = \frac{1}{\pi r^2} \int_{|z-a|\leq r} { f(z) d \lambda_2 }$.
\end{consequence}
\begin{proof}
    $\int_{|z-a| \leq r} { f(z) d \lambda_2 } = \int_{0}^{r} { \int_{0}^{2\pi} {f(a + \rho e^{e\phi}) \rho d \phi } d \rho } = \int_{0}^{r} { 2 \pi f(a) \rho \ d \rho } = 2 \pi f(a) \frac{r^2}{a} = \pi r^2 f(a)$.
\end{proof}

\begin{theorem}
    \textbf{Принцип максимума}.

    $f \in H(\mathbb{C}), \ a \in \Omega$. Если $|f(a)| \geq |f(z)| \ \forall z$ из окр-ти точки $a$, то $f \equiv const$.
\end{theorem}
\begin{proof}
    Пусть $|f(a)| =: M$. Домножим $f$ на $e^{i \alpha}$ так, что $f(a) = M > 0$.

    $|f(a)| = M = \frac{1}{2\pi} \left|\int_{0}^{2\pi} { f(a + r e^{i \phi}) d \phi }\right| \leq \frac{1}{2 \pi} \int_{0}^{2\pi} { |f(a + r e^{i \phi})| d \phi} \leq \frac{1}{2\pi} \int_{0}^{2\pi} { M \ d \phi } = M$.

    Все нер-ва обращаются в равенства $\implies |f(a + r e^{i \phi})| = M \ \forall \phi \ \forall$ маленьких $r$.

    $Re (f(a)) = M = \frac{1}{2\pi} \int_{0}^{2\pi} { Re(f(a + r e^{i \phi})) d \phi } \leq \frac{1}{2 \pi} \int_{0}^{2\pi} { |f(a + r e^{i \phi})| d \phi } \leq M$. Это все равенства $\implies Re (f(a + r e^{i \phi})) = |f(a + r e^{i \phi})| = M \implies f(z) = f(a)$ в окр-ти точки $a \underbrace{\implies}_{\text{т. о единственности}} f(z) \equiv f(a)$.
\end{proof}


\begin{consequence}
    $f \in H(\Omega), \ \Omega$ -- огранич. область, $f \in C(Cl (\Omega))$. Тогда $|f|$ достигает своего $\max$ на границе $\Omega$.
\end{consequence}

\begin{proof}
    $Cl (\Omega)$ -- компакт $|f|$ непрер. на компакте $\implies$ в какой-то точке $a \in Cl (\Omega)$ достигает $\max$.

    Если $a \in \Omega$, то по принципу максимума $f \equiv const$, значит на границе то же самое значение.

    Если $a \not \in \Omega$, то это точка на границе.
\end{proof}



\begin{definition}
    $f \in H(\Omega), \ a \in \Omega, \ a$ -- ноль функции $f$, если $f(a) = 0$.
\end{definition}

\begin{theorem}
    $f \not \equiv 0, \ f \in H(\Omega), \ a \in \Omega, \ f(a) = 0$. Тогда существует $m \in \mathbb{N}$ и $g \in H(\Omega)$, т.ч. $g(a) \not = 0$ и $f(z) = (z - a)^{m} \cdot g(z)$.
\end{theorem}
\begin{proof}
    Разложим $f$ в ряд Тейлора в окр-ти точки $a$.

    $f(z) = \sum_{n=0}^{\infty} { \frac{f^{(n)} (a)}{n!} \cdot (z - a)^{n} }$, $m := \min \{ n : \ f^{(n)} (a) \not = 0 \}$.

    $g(z) = \begin{cases}
        \frac{f(z)}{(z - a)^{m}}, \ z \not = a \\
        \frac{f^{(m)} (a)}{m!}, \ z = a
    \end{cases}$

    $g \in H(\Omega \setminus \{a\})$, $g$ -- непрерывная в точке $a$,  $\implies g \in H(\Omega)$.

    $g(z) = \sum_{n=m}^{\infty} { \frac{f^{(n)} (a)}{n!} (z - a)^{n - m} } \underbrace{\rightarrow}_{z \rightarrow a} \frac{f^{(m)} (a)}{m!}$
\end{proof}












