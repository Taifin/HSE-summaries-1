\Subsection{Голоморфные функции}

Если доказательство не указано, то оно повторяет то, что было в $\mathbb{R}$ (смотреть 1 семестр).

\begin{definition}
    $\Omega$ -- обсласть в $\mathbb{C}, \ f: \Omega \rightarrow \mathbb{C}, \ z_0 \in \Omega$.

    $f$ -- голоморфна в точке $z_0$, если существует $\lim_{z \rightarrow z_0} { \frac{f(z) - f(z_0)}{z - z_0} } =: f'(z_0)$.
\end{definition}
\begin{definition}
    $f$ комплексно дифф. в точке $z_0$, если $\exists k \in \mathbb{C}$:

    $f(z) = f(z_0) + k (z - z_0) + o(z - z_0)$ при $z \rightarrow z_0$.
\end{definition}

\begin{statement}
    $f$ -- голоморфна в точке $z_0 \Leftrightarrow f$ комплексно дифф. в точке $z_0$ и $k = f'(z_0)$.
\end{statement}

\begin{consequence}
    $f$ и $g$ голоморфны в точке $z_0$. Тогда 

    \begin{enumerate}
        \item {
            $f \pm g$ голом. в точке $z_0$
        }
        \item {
            $f \cdot g$ голом. в точке $z_0$
        }
        \item {
            Если $g(z_0 \not = 0)$, то $\frac{f}{g}$ голом. в точке $z_0$.
        }
        \item {
            Если $h$ голом. в точке $f(z_0)$, то $h \circ f$ голом. в точке $z_0$.
        }
    \end{enumerate}
\end{consequence}

\begin{remark}
    $f: \Omega \rightarrow \mathbb{C}$

    $z = x + iy, \ f(z) = f(x + iy) = g(x + i y) + i h(x + iy): \ g, h : \Omega \rightarrow \mathbb{R}$.

    $\frac{\delta f}{\delta x} (z_0) = \lim_{h \rightarrow 0, \ h \in \mathbb{R}} {\frac{f(z_0 + h) - f(z_0)}{h}} = f'(z_0)$.

    
    $\frac{\delta f}{\delta y} (z_0) = \lim_{h \rightarrow 0, \ h \in \mathbb{R}} {\frac{f(z_0 + i h) - f(z_0)}{h}} = \frac{f'(z_0)}{i} = -i \cdot f'(z_0)$.
\end{remark}

\begin{remark}
    $\binom{g(x + iy)}{h(x + iy)} = \binom{g(x_0 + i y_0)}{h(x_0 + iy_0)} + \binom{a \ b}{c \ d} \binom{a - x_0}{y - y_0} + o(|| (x - x_0, y - y_0) ||)$.

    $k = \alpha + i \beta$

    $k \cdot (z - z_0) = (\alpha + i \beta) ( (x - x_0) + i (y - y_0)) = \alpha(x - x_0) - \beta(y - y_0) + i (\beta (x - x_0) + \alpha (y - y_0))$

    Вещественная линейность + $\binom{\alpha \ -\beta}{\beta \ \alpha} \Leftrightarrow$ комплескная линейность.
\end{remark}



